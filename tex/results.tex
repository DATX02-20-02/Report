\chapter{Results}

% Explain Like I'm 5
% Mention CityCraft on first page of introduction and in abstract as well.
This chapter presents CityCraft, the standalone city generation software that was implemented as a result of this research.
CityCraft is a graphical desktop application that lets users interactively generate 3D cities, which can be exported as glTF files and used in third-party modeling software such as Blender.
The application combines computer graphics theory and several PCG techniques to achieve its results.
An example of CityCraft in use is shown in Figure \ref{fig:screenshot}, and a CityCraft generated model is shown rendered in Blender in Figure \ref{fig:blender}.

\begin{figure}[H]
  \centering
  \includegraphics[width=\textwidth]{figure/results/screenshot.png}

  \caption{A screenshot taken from CityCraft as a user generates a city. In the top-right resides a menu panel which the user operates to perform the generation. The view of the world is control through a camera which is rotated with the mouse and moved with the keyboard.}
  \label{fig:screenshot}
\end{figure}

\begin{figure}[H]
  \centering
  \includegraphics[width=\textwidth]{figure/results/screenshot.png}

  \caption{A city model exported from CityCraft and rendered inside Blender.}
  \label{fig:blender}
\end{figure}

CityCraft's GUI follows the \textit{Wizard} design pattern \cite{yer_a_wizard} to guide its users through the generation process, which is split into four steps (see Figure \ref{fig:guisteps}).
The four generation steps are terrain, roads, streets, and cities.
The last of which performs the generation of buildings, parks, and parking lots. 

\begin{figure}[H]
  \centering
  \begin{subfigure}[b]{0.24\textwidth}
    \includegraphics[width=\textwidth]{figure/results/gui1.png}
  \end{subfigure}
  \begin{subfigure}[b]{0.24\textwidth}
    \includegraphics[width=\textwidth]{figure/results/gui2.png}
  \end{subfigure}
  \begin{subfigure}[b]{0.24\textwidth}
    \includegraphics[width=\textwidth]{figure/results/gui3.png}
  \end{subfigure}
  \begin{subfigure}[b]{0.24\textwidth}
    \includegraphics[width=\textwidth]{figure/results/gui4.png}
  \end{subfigure}

  \caption{CityCraft's four steps of generation. Each step has a separate settings panel and each generation can be performed multiple times.}
  \label{fig:guisteps}
\end{figure}

The four generation steps of the GUI are implemented as eight PCG-based generators, all of which are orchestrated by a single module called the \textit{WorldGenerator}.
The following subchapters will go into detail about the results that each of these generators produces and how their results contribute to the final city models.

\section{Terrain Generation}

When users start CityCraft, they are first presented with an endless ocean and a menu panel, as shown in Figure \ref{fig:no_terr}.
This is the start of the generation process, and the ocean represents the blank canvas on which the rest of the world will be built.

\begin{figure}[H]
  \centering

  \includegraphics[width=0.7\textwidth]{figure/terrain_not_generated.png}
  \caption{The application state before the \textit{Generate Terrain} button has been pressed. The ocean is visible from the very beginning.}

  \label{fig:no_terr}
\end{figure}

The very first step of the generation processs is to generate the terrain
, whose settings are adjusted in the top-right menu panel.
The terrain settings that the user can adjust are:
\begin{easylist}
  @ Sea level: modifies the water level.
  @ X/Z offset: offsets the sampling location along respective axis of the noise function, effectively changing the height values of the terrain.
  @ Width/Depth: adjusts the size of the entire terrain.
\end{easylist}

The terrain is generated by creating a surface mesh whose vertices are given height values sampled from a 4-layered Simplex noise function.
The terrain utilizes a single texture, which is repeated several times with randomized UV coordinates.
The terrain is also colored based on height values, forming snowy mountain tops and green valleys.
An example of a generated terrain is shown in Figure \ref{fig:terr}.

\begin{figure}[H]
  \centering

  \includegraphics[width=0.7\textwidth]{figure/terrain_generated.png}
  \caption{The application state after \textit{Generate Terrain} has been pressed. The user may change settings and regenerate the terrain as many times as they like.}

  \label{fig:terr}
\end{figure}

The ocean was just implemented as a large textured plane that clips through the terrain, forming lakes at its intersections.


\section{Population Generation}

When the terrain has been generated, the PopulationGenerator gives the user a visual tool to place a city marker.
When one or more markers are placed, they will remain in their designated positions until interacted with by the user, or when the user presses the Generate Roads button.

A placed city marker can be interacted with by hovering over it, which will turn it red.
Clicking on it will make it yellow, which shows that it is selected. 

\begin{figure}[h!]
  \centering

  \includegraphics[width=0.8\textwidth]{figure/citymarkers.png}
  \caption{The city marker tool visuals. }

  \label{fig:citymarkers}
\end{figure}

The selected city marker has multiple options:

It can be moved by clicking again and dragging the marker around,
it can be resized by scrolling up and down, increasing and decreasing the effective radius of the circle,
and it can be deleted by pressing the Delete key on the keyboard.

Road generation begins after pressing the generate roads button.
Before the roads are generated, a population density map is generated which covers the entire terrain.
Then, the city marker amplifiers add density to the base noise generated in the spots where it was placed.

Roads and highways will tend towards densely populated areas, and the size of buildings will vary depending on how intense the density is.

\subsection{Road Generation}

After the city markers are specified and the population density map has been created, it is time to generate the road network.
Using the city marker, the road generator creates its Agents in the area specified by the markers and assigns them strategies depending on city type for that marker.
The application has Agent factories and strategies for Paris and Manhattan cities, and an example of these are shown in Figure~\ref{fig:results_city_paris} and Figure~\ref{fig:results_city_manhattan}.

The Paris city have clear, distinct rings around the center of the town, while the Manhattan city have multiple, long straight roads.
Both cities then have streets branching out from their main roads, which in turn also branch out from themselves to form a grid-like street neighborhood.

\begin{figure}[H]
  \centering
  % Use two minipages to add padding for the figure and its caption
  \begin{minipage}{.45\textwidth}
    \centering
    \begin{minipage}{.9\textwidth}
      \centering
      \includegraphics[width=\textwidth]{figure/results/city_paris.png}
      \caption{Example of a fully generated Paris-style city.}
      \label{fig:results_city_paris}
    \end{minipage}
  \end{minipage}
  \begin{minipage}{.45\textwidth}
    \begin{minipage}{.9\textwidth}
      \centering
      \centering
      \includegraphics[width=\textwidth]{figure/results/city_manhattan.png}
      \caption{Example of a fully generated Manhattan-style city.}
      \label{fig:results_city_manhattan}
    \end{minipage}
  \end{minipage}
\end{figure}

Both city types uses the population map for different decisions that the Agents makes during the course of their lifetime.
Roads outside the bounds of the city navigates towards higher density areas while roads inside the bounds are shaped to some degree according to the population map.
For example, the streets are not always created when the population is too low.

\section{City Block Generation}

Once roads and streets have been generated, the city block generation starts.
This generation step produces a list of polygons that mark which areas of the terrain are suitable for buildings, parks, and parking lots.
City blocks may vary significantly in size, but a limit is set on how large they can become (see Figure \ref{fig:results_blockgen1}).

\begin{figure}[H]
  \centering

  \includegraphics[width=0.8\textwidth]{figure/results/blockgen1.png}
  \caption{Generation of city blocks in a road network, where pink lines outline city blocks. Notice how the largest areas are not treated as blocks.}

  \label{fig:results_blockgen1}
\end{figure}

Each city block is also guaranteed to be connected to the road network, such that roads, streets, or both surround it.
Consequently, each block has to be slightly inset in order to make room for the road meshes.
This necessity becomes more apparent when all meshes are rendered (see Figure \ref{fig:results_blockgen2}).

\begin{figure}[H]
  \centering

  \includegraphics[width=0.8\textwidth]{figure/results/blockgen2.png}
  \caption{Blocks generated on a 3D terrain. Notice how each block is connected to the road network. The colored lines are only shown during development.}

  \label{fig:results_blockgen2}
\end{figure}

Each city block is assigned a label that helps the subsequent plot generation step determine what to generate inside each city block.
In Figure \ref{fig:results_blockgen2} these labels are visualized as green triangles, meaning they have not been processed yet.
The green triangles are also used to mark blocks that became too small after being inset, effectively excluding them from further processing.
An example of this behavior is shown in Figure \ref{fig:results_blockgen3}.

\begin{figure}[h!]
  \centering
  \begin{subfigure}[b]{0.469\textwidth}
    \includegraphics[width=\textwidth]{figure/results/blockgen3.png}
  \end{subfigure}
  \quad
  \begin{subfigure}[b]{0.471\textwidth}
    \includegraphics[width=\textwidth]{figure/results/blockgen4.png}
  \end{subfigure}

  \caption{City blocks with different labels (shown as colored squares) shown without (left) and with (right) buildings rendered. Red labels indicate parks, while the other colors shown in this figure indicate various types of buildings.}
  \label{fig:results_blockgen3}
\end{figure}

The implemented labels are \textit{industrial}, \textit{suburbs}, \textit{downtown}, \textit{skyscrapers}, \textit{apartments} and \textit{parks}.
Each label weights what type of content should be generated so that all plots within the same block share a certain theme.
Finally, once the blocks are generated, they are passed on to the plot generation.
\section{Plot Generation}
Each block that is sent into the PlotGenerator is treated differently depending on what the block label is. 
If the block label is either \textit{Parks} or \textit{Parking}, then the PlotGenerator does not split the block. 
The entire block is turned into a plot and is sent to ParkGenerator or ParkingGenerator respecitvely.
The rest of the block labels are split into n parts, where n depends on the area of the block and some randomness. 

Figure~\ref{fig:plot} and \ref{fig:plot2} demonstrates some example outputs from the PlotGenerator.

\begin{figure}[H]
  \centering

  \includegraphics[width=0.8\textwidth]{figure/plot2.png}
  \caption{Close-up of the plot splitting algorithm.}

  \label{fig:plot2}
\end{figure}

\begin{figure}[H]
  \centering

  \includegraphics[width=0.8\textwidth]{figure/plot.png}
  \caption{Generated plots within a larger city.}

  \label{fig:plot}
\end{figure}

Each plot is assigned a plot label, which is later used for the PlotContentGenerator to determine what should be generated within the plot.
The plot labels are \textit{Manhattan}, \textit{Skyscraper}, \textit{Park}, \textit{Parking}, and \textit{Empty}.
\textit{Manhattan} and \textit{Skyscraper} are two different sub-generators for BuildingGenerator. 
\textit{Park} and \textit{Parking} are used by the ParkGenerator and the ParkingGenerator respectively.
If the plot label is \textit{Empty}, then no content is generated upon the plot. 

\section{Building Generation}

\begin{figure}[H]
  \centering

  \includegraphics[width=0.7\textwidth]{figure/skyline.PNG}
  \caption{Skyline of \textit{Manhattan} and \textit{Skyscraper} buildings together.}

  \label{fig:skyline-result}
\end{figure}


The Building Generator creates buildings for the plots whose labels are \textit{Manhattan} or \textit{Skyscraper}. 
An example of a skyline with both \textit{Manhattan} and \textit{Skyscraper} can be seen in Figure \ref{fig:skyline-result}.
\textit{Manhattan} buildings are very versatile via the possibility to have different floor types L-systems and different wall segment L-systems. 
The final implementation had a single floor type generator called \textit{StraightManhattanFloorsGenerator (Straight)}. 
The name was derived from the possibility of having buildings that became smaller as the building grew higher. 
This was never implemented, though. 

\textit{Straight} can generate four different floor types: \textit{First}, \textit{Normal}, \textit{EveryOther}, and \textit{RepeatWindow}. 
An L-system is used to generate the floor types. The axiom is \textit{First} and one of \textit{Normal}, \textit{EveryOther}, or \textit{RepeatWindow}. 
The generation is quite simple. It just repeats one of the last three mentioned floor types until it has reached its desired height. 

The different floor types mentioned represent an L-system that generates some wall segments for the floor per walls. 
First can generate \textit{Corner}, \textit{ShopWindow}, \textit{Wall}, and \textit{Door}. \textit{Normal} can generate \textit{Corner}, \textit{Wall}, and \textit{Window}, but it only generates half of the wall. 
For the other half, it copies the first half and reverses it. \textit{EveryOther} can generate \textit{Corner}, \textit{Wall}, and \textit{Window}. 
It switches though between \textit{Wall} and \textit{Window}. The last floor type, \textit{RepeatWindow}, can only generate \textit{Corner} and \textit{Window}. 
It only has windows between the corners. An example of the generation of \textit{Straight}, with the four different floor types, can be found in Figure \ref{fig:wall-segment-generator}.


\begin{figure}[H]
  \centering

  \begin{subfigure}[b]{0.25\textwidth}
    \includegraphics[width=\textwidth]{figure/building-every-other.PNG}
    \caption{\textit{EveryOtherFloor}.}
  \end{subfigure}
  \quad
  \begin{subfigure}[b]{0.25\textwidth}
    \includegraphics[width=\textwidth]{figure/building-normal.PNG}
    \caption{\textit{MirrorFloor}.}
  \end{subfigure}
  \quad
  \begin{subfigure}[b]{0.25\textwidth}
      \includegraphics[width=\textwidth]{figure/building-only-window.PNG}
      \caption{\textit{OnlyWindowFloor}.}
  \end{subfigure}
  
  \caption{The three different wall segment generators. Notice that each building has a \textit{FirstFloor} wall segment generator on the first floor.}
  \label{fig:wall-segment-generator}
\end{figure}

Skyscraper has two different textures from which different sub-regions were sampled. In Figure \ref{fig:skyscraper-result}, you can see examples of them. 


  \begin{figure}[H]
    \centering
  
    \begin{subfigure}[b]{0.45\textwidth}
      \includegraphics[width=\textwidth]{figure/skyscraper-close-up.PNG}
    \end{subfigure}
    \quad
    \begin{subfigure}[b]{0.45\textwidth}
      \includegraphics[width=\textwidth]{figure/wack.PNG}
    \end{subfigure}
    
    \caption{Various \texit{Skyscraper}s.}
    \label{fig:skyscraper-result}
  \end{figure}
 
\section{Park Generation}

Once the plot generation is finished, the park generation starts working towards filling some of the generated plots with parks.

The parks are generated through two steps, the first step is to generate a natural-looking path, and the second step is to fill the park with objects such as trees, bushes, and rocks. 
The algorithm for paths works by firstly generating a random point on one of the edges of the plot and then scattering points randomly, although with a minimum distance from one another, across the plot. 
Secondly, the points are sorted by distance, ensuring that the path always traverses to the closest possible point. 
Afterward, once all points are found, the algorithm either creates a path that loops back to the starting point, or a path towards another random point on the edge of the plot (see Figure~\ref{fig:parks}).
Finally, exits/entries are added to the park by creating more paths to the edges of the plot.  

The algorithm for object placement runs after the one for creating paths as the objects have to consider the coordinates of the path.
The first step of the object placement algorithm is to assign each object-type with an object radius, indicating the distance at which other objects are allowed to be created. 
Afterward, a probability of being generated is assigned to each object type.
The algorithm then runs several times, and each time an object-type is picked, a model of that object-type is randomly chosen, scaled, and rotated before being generated and placed somewhere in the park.

\begin{figure}[H]
  \centering
  \begin{subfigure}[b]{0.4\textwidth}
    \frame{\includegraphics[width=\textwidth]{figure/loopy}}
    \caption{Park with looping path.}
  \end{subfigure}
  \quad
  \begin{subfigure}[b]{0.4\textwidth}
    \frame{\includegraphics[width=\textwidth]{figure/path}}
    \caption{Park with path that does not loop.}
  \end{subfigure}
  \caption{Two example of different generated parks. One with a path that loops back, and one that creates a path to another edge of the plot.}
  \label{fig:parks}
\end{figure}
 
It is worth noting that even though the paths always end up looping, or connecting to a random edge, their shapes still vary a lot, leading to quite interesting results (see Figure~\ref{fig:park_ex2}).
The fact that object placement takes the path into account also makes the park layout appear less random and more relative to its environment. 
This provides a more consistent feeling to the different parks.
 
\begin{figure}[H]
  \centering
  \begin{subfigure}[b]{0.4\textwidth}
    \frame{\includegraphics[width=\textwidth]{figure/loopytwo}}
  \end{subfigure}
  \quad
  \begin{subfigure}[b]{0.4\textwidth}
    \frame{\includegraphics[width=\textwidth]{figure/nonlooptwo}}
  \end{subfigure}
  \caption{Two additional examples, showcasing different results.}
  \label{fig:park_ex2}
\end{figure}
 

\section{Parking Lot Generation}

Once the plot generation is finished the parking lot generation starts to fill some of the generated plots with parking lots.
The generated parking lots consist of one or more rows of parking spaces (see Figure~\ref{fig:results_parking_sizebased}).

\begin{figure}[H]
   \centering
   \begin{subfigure}[b]{0.38\textwidth}
     \frame{\includegraphics[width=\textwidth]{figure/results/parking/bigplot}}
     \caption{Large parking lot consisting of two rows.}
   \end{subfigure}
   \quad
   \begin{subfigure}[b]{0.52\textwidth}
     \frame{\includegraphics[width=\textwidth]{figure/results/parking/smallplots}}
     \caption{Two small parking lots consisting of one row.}
   \end{subfigure}
     \caption{Two examples of different sized parking lots created by the generator.}
   \label{fig:results_parking_sizebased}
 \end{figure}

These parking lots are generated by approximating the largest possible rectangle that fits inside a given plot, and then generating parking lots inside of it.
The number of rows is based on the size of the computed rectangle, and the algorithm aims to fit as many rows of parking lots inside the rectangle such that there still exists space between the rows for cars to enter.

Having the parking lots consist of multiple rows depending on the size gives some more variety, but the decision to do this was also based on real-world parking lots (see Figure~\ref{fig:parkings}).
In the left example in Figure~\ref{fig:parkings}, a road is used to separate parking spaces, making it impossible for parked cars to be surrounded and stuck by other parked cars.
The right example, however, has a more interesting shape, the parking lot in the center of it is shaped in the two-row style, while the surrounding rectangle is large enough that cars can easily drive around and not be blocked by parked cars.
This type of parking space is difficult to generate reliably in arbitrary plot shapes, consequently shapes like these were left out of the scope for this project.
