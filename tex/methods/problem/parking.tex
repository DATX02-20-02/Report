\subsection{Parking Lot Generation}
The ParkingGenerator is implemented as a function that takes a plot and its underlying terrain as input and produces a parking lot as output (see Table~\ref{table:parking}).
This generator in particular is responsible for filling roughly ten percent of the generated cities with parking lots, a percentage based on research conducted in Phoenix,~AZ~\cite{parking_percent}.
\begin{table}[H]
   \centering
   \begin{tabular}{lllll}
     \textbf{Input}                           &               & \textbf{Function}            &               & \textbf{Output}         \\
     \midrule
     \textit{Plot, Terrain}                   & $\rightarrow$ & \textbf{ParkingGenerator}       & $\rightarrow$ & \textit{Parking lot}           \\
     \bottomrule
   \end{tabular}

   \caption{Definition of the ParkingGenerator function which is responsible for generating parking lots.}
   \label{table:parking}
 \end{table}
 \vspace{-0.4cm}

One considered approach for generating parking lots was Esri CityEngine's \cite{Esri}.  % having a hard time finding this source, Anton who found it... plz help me. 
What they use, for at least some of their generation is simply covering areas with textures. This can be seen in their parks and parking lots. 
This was at first considered to be a viable option for the project, only it was found that it offers little alternatives for modification. 

Modification, in this remark, includes the shape of the entire parking lot as well as the size of the individual parking spaces.
As every other generator provides fully scalable content, it would be inconvenient for the ParkingGenerator to be any different.
If it were to make use of quads with a single texture applied to them, it would remove the possibility of scaling the size of the parking spaces within the quad without scaling the quad itself.

The approach instead elected for was an algorithm that generates the parking spaces individually.
The main benefit of this approach was that it tackled the issue with the previously mentioned approach, which was scaling.
This approach instead opted to generate the individual parking spaces so that they would not be limited to the size provided by a texture, this in turn offered more control when scaling the parking lots to fit with any other generated content.

The algorithm constructed for this approach works by first using a function for approximating the largest rectangle inside any given polygon.
Afterward, based on the size of the rectangle, the algorithm then generates either two or four columns of parking spaces (see Figure~\ref{fig:sizebased}).
How this works is that the algorithm uses the corners of the rectangle as reference points.
These reference points are then inset to avoid creating parking spaces outside of the rectangle, thereafter the algorithm creates white stripes (representing parking lines) from these inset reference points.
The generation of these lines are implemented as two parallel processes. 
The first process works from the top down to the middle and the second process works from the bottom up to the middle, this is done by simply increasing the distance from the previous point with a constant offset. 
The size of this offset is supposed to approximately represent the size of a theoretical car in our application.
What type of parking lot to be generated is determined is by judging whether four columns with a gap between them (representing a road within the parking lot) fits inside the approximated rectangle or not. 
The generated road was decided to be twice the size of the offset, this decision was made based on the fact that parking lots are rarely designed to only be driven on one-way.
This road is generated by simply finishing the algorithm one step earlier, resulting in the last four parking lots not being generated. 

The reasoning behind this algorithm was the group's observation that parking lots seem to have a rectangular shape generally (see Figure~\ref{fig:parkings}).
This shape does of course not apply to every parking lot in the world, however attempting to create parking lots of any size and shape would require a lot more effort, and was therefore deemed out of scope for this project. 
The decision to have a road between four columns of parking lots is also based on real-world parking lots and is visible in the left image in Figure~\ref{fig:parkings}. 
Without this road, cars would not be able to leave or enter the parking lot if enough cars are occupying it.

\begin{figure}[H]
  \centering
  \begin{subfigure}[b]{0.55\textwidth}
    \frame{\includegraphics[width=\textwidth]{figure/parking1}}
  \end{subfigure}
  \quad
  \begin{subfigure}[b]{0.395\textwidth}
    \frame{\includegraphics[width=\textwidth]{figure/parking2}}
  \end{subfigure}
  \caption{Two examples of parking lots observed by the project group, showcasing the rectangular shapes mentioned above.}
  \label{fig:parkings}
\end{figure}

\begin{figure}[H]
  \centering
  \begin{subfigure}[b]{0.485\textwidth}
    \frame{\includegraphics[width=\textwidth]{figure/fourcol}}
    \caption{Large parking lot consisting of four columns.}
  \end{subfigure}
  \quad
  \begin{subfigure}[b]{0.45\textwidth}
    \frame{\includegraphics[width=\textwidth]{figure/twocol}}
    \caption{Small parking lot consisting of two columns.}
  \end{subfigure}
    \caption{Two examples of different sized parking lots created by the generator.}
  \label{fig:sizebased}
\end{figure}

