\subsection{Plot Generation}  
A plot in this project was similarly defined with city block, where the difference is that a block can contain multiple plots. 
A plot can also have different labels with what kind of content that is generated within it. 
The label of the block, along with the population for each plot, is the deciding factor when plot labels is selected. 

\begin{table}[H]
    \centering
    \begin{tabular}{lllll}
      \textbf{Input}                           &               & \textbf{Function}            &               & \textbf{Output}         \\
      \midrule
      \textit{Block, PopulationMap}            & $\rightarrow$ & \textbf{PlotGenerator}       & $\rightarrow$ & \textit{Plot[]}         \\
      \bottomrule
    \end{tabular}

    \caption{Definition of the PlotGenerator function, which is responsible to split a block into one or more plot.}
    \label{table:plotgen}
  \end{table}
  \vspace{-0.4cm} 

PlotGenerator is needed because the blocks that are created often are too large to house, for example, a single building.
Instead of placing multiple buildings on a single block, using multiple plots leads to greater modularity to, for example, mix and match different plot labels on a single block.
Smaller plots mean less responsibility for the different content generators. 
They have the sole responsibility to fill their plot of land with whatever they like. 
Having different plots also grants the ability to have a large variety of plot labels within a single block. 
Something that can be quite complex if there is no clear separation.

Splitting a block into plots is essentially about dividing a polygon sensibly. 
Both block and plot are in the end a continuous area with defined points, the same as a polygon. 
Note that block size and population are not a part of splitting the polygons, but merely how the plot label is decided. 

One of the proposed solutions was to draw a line through the given polygon and let the new unique sub polygons be the result. 
This would have been a straight line going through the block. 
This does not, however, result in a good split necessary. 
It is hard to control the size of the polygons, as some can become quite small. 
There’s also the issue with concave polygons where it can become hard to create visually pleasing plots with consistency. 
There is also the issue where you cannot let the number of plots be a deciding factor to find the needed cut easily. 
This kind of algorithm would be outside of our scope of this project. 

% Add figures explaing.
An article about splitting polygon was found that explained how to split a polygon into N parts where each part would have equal size\cite{polygon_splitting_article}. 
In this article, you can find a more in depth explanation of the article. 
Each iteration finds possible cuts that could be made. 
The cuts are found by finding different ways to draw lines through what is left of the polygon. 
Some cuts are omitted in cases where concave polygons exist since they can be outside of the polygon. 
The cut that is closest to the size for each sub polygon is the one that is applied and added to the output. 
This was ultimately the algorithm used in the final product. 

As previously mentioned, block labels and population are factors when deciding the label for each plot. 
If the block label is, for example, a park or a parking lot, then the entire block is converted to one plot with the same corresponding plot label. 
However, if the block label is \textit{buildings}, then also the population with some randomness plays a big part in what label of the building is selected. 
If the population is high, then it is more likely that the plot label is a \textit{skyscraper} to accommodate the population. 
