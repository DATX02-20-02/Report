\subsection{Plot Generation}  

In this project, a plot is defined as a continuous area of land within a block that is suitable for containing a single building, park, or parking lot.
The generation of such plots is managed by PlotGenerator (see Table \ref{table:plotgen}).

\begin{table}[H]
  \centering
  \begin{tabular}{lllll}
    \textbf{Input}                           &               & \textbf{Function}            &               & \textbf{Output}         \\
    \midrule
    \textit{Block, PopulationMap}            & $\rightarrow$ & \textbf{PlotGenerator}       & $\rightarrow$ & \textit{Plot[]}         \\
    \bottomrule
  \end{tabular}

  \caption{Definition of the PlotGenerator function, which is responsible to split a block into one or more plot.}
  \label{table:plotgen}
\end{table}
\vspace{-0.4cm} 

Each plot also has a label associated with it that determines what type of content should later be generated inside it. 
The label of the block, along with the population for each plot, is the deciding factor when plot labels are selected. 
For instance, if the block label is \textit{Park} or \textit{Parking}, then the entire block is converted to one plot with the corresponding plot label. 
However, if the block label is \textit{Downtown}, then the population density is used to sample a building from a pool of building types. 
As the population increases, so does the probability that the plot label will be set to \textit{Skyscraper}, to accommodate for the population. 

PlotGenerator was designed because blocks tended to be too large to suit a single structure, such as a building.
Instead of directly placing multiple buildings on a single block, the block is divided into multiple plots.
This separation of concern leads to greater modularity and makes it easier to mix a large variety of different plot labels within a single block.
The core of this plot division problem was essentially about splitting an arbitrary polygon into sensible sub-polygons. 

One of the proposed solutions was to draw a straight line through the block polygon and let the new sub-polygons be the result. 
This does not, however, result in a good split necessary. 
It is hard to control the size of the polygons, as some can become quite small. 
There is also the issue with concave polygons, where it can become difficult to create visually pleasing plots with consistency. 
One can not either reliably let the number of plots be a deciding factor to find the needed cut. 

% Add figures explaing.
An article about splitting polygons was found that explained how to cut a polygon into $N$ parts, where each part would have equal size \cite{polygon_splitting_article}. 
In this algorithm, each iteration finds possible cuts that could be made. 
The cuts are found by finding different ways to draw lines through what is left of the polygon. 
Some cuts are omitted in cases where concave polygons exist since they can be outside of the polygon. 
The cut that is the shortest for each sub-polygon is the one that is applied and added to the output. 
This was ultimately the algorithm used in the final product. 
