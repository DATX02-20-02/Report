\subsection{Road Generation}
The most basic aspect of a city that describes the overall city structure is the
road network. As such, it was important to get this right early on in the
development of the city generator. The application needed a method that was both
flexible and also offered realistic looking results.

At least two methods were considered when designing the road generator. One of
them were based on a recursive approach where the world is divided into multiple
cells and roads were placed between these cells. However, this type of algorithm
did not provide realistic looking results and, while being quite flexible, was
overly complex for creating a good road network.

Instead, the algorithm that was chosen was an Agent-based solution that works by
simulating road workers, which will be referred to as Agents, which only goal is
walking around the world depending on certain strategies and creating roads.
Agents could also decide to branch into multiple, new Agents, creating
intersections in the road network.
This algorithm resulted in realistic and good looking results and also offered
lots of flexibility. The goal of this chapter is describing how this method was
used to generate the cities in the application.

\subsubsection{Agent Strategies}
This section is dedicated to describing exactly what a strategy is. The use of
strategies means that Agents are considered to be ``dumb'', and can only move
around when it is told where to move. Which strategy to use depends entirely on
what the goal of the generation is.

Strategies also defines the configuration of the Agent, such as step size, how
many steps an Agent can take before terminating and how many times an Agent can
branch. The strategy that the Agent uses is responsible for deciding when an
Agent should terminate, except for step count which is handled automatically for
all strategies.

\subsubsection{Plotting Cities}
The algorithm requires some inputs in order to get started, and that consists of
a city type and a position. Then, depending on the city type, some other
variables are needed; Paris-like cities need a radius in order to define the
rough size of the resulting city, while Manhattan-like cities need a rough width
and height.

In the GUI of the application, the user needs to place down city markers in the
world with the desired position and size, then the generator can be started
which creates Agents that will act according to the city type that was chosen by
the user.

\subsubsection{Agent Road Generation}
The first step of the generation is to create a rough structure of the road
network. This includes the main roads of the city that is commonly found near
the center and boundaries of cities, but also highways outside the city.

The generation for main roads are different depending on the city strategy. For
example, Paris-like cities have distinct rings within the city boundaries, while
Manhattan-like cities generally follow a more organized and grid-like structure.
The generation of the main roads should mimic these characteristics.

So far, Paris and Manhattan-like cities have been described, but the algorithm
can be tuned to simulate other types of cities, the only thing that is needed is
a suitable strategy fo the Agents to follow. For example, if one wanted to
generate a chaotic strategy without any sensible rules, one could implement a
strategy that would make the Agents walk completely randomly.

Furthermore, Agents can switch strategy mid-generation, be it randomly or
depending on some variables that the Agent has access to. One example of this
might be that if an Agent detects that there is a relatively low population
density, it might decide to switch to an industrial type of strategy, which
could make multiple long, parallel lines of roads which is often found in
industrial areas in Sweden.

\subsubsection{Highway Generation}
The algorithm aims to create highways outside the boundaries of the cities,
which has a strategy that makes Agents prefer moving towards higher population
density. This type of strategy resulted in the tendency that connections between
cities are created.

Highways also have a lower probability of branching into more agents, which
resulted in roads that are longer with less exits, mimicking what can be seen in
the real world.
