\section{Process and Workflow}

This chapter details how the group collaborated throughout the project.
The first subchapter describes how the group approached the software development process and how the group distributed the workload.
The second subchapter describes the tools and software used to aid the group to collaborate more effectively.

\subsection{Software Development Process}

The group decided that the development of the application would follow an agile development process.
Agile software development generally involves self-organized teams working in iterations, where requirements and solutions actively evolve throughout the development process~\cite{agile101}.
This approach felt suitable for the project's loose definition of when the application could be considered complete.
For this reason, the group decided to work in intervals of two-week long iterations.
At the end of each iteration, the previous iteration was evaluated and new goals were defined for the next iteration.

Group meetings were the main means of discussing the status of the project, general improvements, and other topics related to the project.
All meetings were considered to be mandatory and therefore followed an opt-out regimen.
This included weekly meetings with the project supervisor.

Initially, the group agreed to meet twice a week for meetings.
The group also decided to meet twice a week for mutual work sessions.
The work sessions were not mandatory, but the group still enforced an opt-out regimen to encourage frequent collaboration.
The intended reason for these sessions was to help each other solve problems that often required communication between the different implementations.
This approach was later discarded in favor of a fully remote workflow due to social distancing guidelines.

After transitioning to a remote workflow, virtual meetings were held three times a week instead.
The remote meetings followed the same arrangement as before, but included a formal stand-up meeting structure.
This addition attempted to further coordinate and track the progress of each group member.
During each stand-up, every group member answered the following questions related to the iteration goal.

\begin{easylist}
  @ What did I complete since the last meeting?
  @ What do I plan on finishing until the next meeting?
  @ Do I see any obstacles that could hinder me or the rest of the team?
\end{easylist}

Since all members worked on the same codebase, a certain level of correctness and readability had to be imposed.
Primarily, the group addressed these concerns by requiring code reviews, enforcing CI pass status, and by writing a Definition of Done (DoD) document.
The usage of the DoD involved verifying that any submitted code followed a certain checklist of requirements.
For instance, it was required that submitted code would not throw any warnings or errors in the Unity Editor. 

% When dividing up the workload between group members, each generator described in Chapter~\ref{sec:city-gen-arch} facilitated a natural division of work since a defined interface of the data exchanged between the generators had been established early on.
% NOTE: This feels very discussion-like.
% This way of splitting up the workload between generators proved both beneficial and disadvantageous in some regard.
% For one, each group member could initially regard the assigned generator as a simple, confined task that could be developed in isolation.
% This allowed group members to implement different parts of the software simultaneously without any major conflicts or bottlenecks.
% For example, the first iteration of the road generator only considered a flat terrain seeing that the terrain generator was not fully implemented yet.

\subsection{Collaboration Tools}

Version control was managed using Git~\cite{git} and hosted on GitHub~\cite{github} for collaboration purposes.
This approach was well suited for the application's non-linear workflow and the need for team members to work locally on different computers.

The correctness of requested code changes to version control was also alleviated with the help of Continuous Integration~(CI).
CI is the process of automating the build and testing of code every time a developer commits changes to version control.
The project group made use of two CI tools for validating committed code.
One such tool compiled the code and reported compilation errors on every requested change.
This made it easy for other reviewers to quickly see whether the requested changes were syntactically legal.
The second tool analyzed the code and reported stylistic deviations from what the group had defined in the tool's configuration file.
This tool could also directly reformat the code according to the specified format which helped maintain readability since the codebase was consistently formatted.

Administrative documents were stored in a shared folder hosted on Google Drive~\cite{google_drive}.
This includes documents related to meetings, group contracts, Definition of Done, and progress updates.

Communication within the group was accomplished through the online messaging service Slack~\cite{slack}.
For conducting meetings, video communication applications such as Discord~\cite{discord} and Zoom~\cite{zoom} were used.

