\subsection{Framework}

Both GUI and 3D graphics would inevitably be an aspect of this research but they were not seen as the core parts.
Therefore, a 3D library, framework, or engine was needed to leverage the workload on this front.
LWJGL \cite{lwjgl}, JMonkeyEngine \cite{jmonkey}, Unity \cite{unity}, Unreal Engine \cite{unreal}, and Godot \cite{godot} were all considered in this project.
Unreal Engine was excluded because of its steep learning curve, and the way it favors visual programming using something called \textit{blueprints}.
Godot was excluded for its limitations in 3D and its young and small community.
LWJGL and JMonkeyEngine were omitted for the amount of work that would have been required to implement simple features like visual debugging and live recompilation.

The choice ended up being the Unity 3D engine because of its large community, extensive range of tutorials, concise documentation, stability, cross-platform support, and wide array of first- and third-party tools.
Moreover, a few group members already had previous experience with this engine, so this choice also included fewer uncertainties than the other options did.

At the time, Unity officially only supported C\#, so the choice of programming language became straightforward.
This aspect was taken into consideration when choosing the engine, and did not occur as a concern since all project members had previous experience with the Java programming language, which is similar to C\#.

Thus, the application would be designed and debugged from the Unity editor, but it also had to be able to run outside of the editor as a standalone program.
Consequently, the application could not depend on functionality provided in Unity's editor to work.
This restriction is normally not a problem, but it did complicate the serialization of runtime objects into standard 3D file formats, which was needed to export the city models.