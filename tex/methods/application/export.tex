\subsection{Exportation of Cities as 3D Models}

The cities had to be exported into some widely used format, and this had to be done through a library for the process to be stable and to save time.
For this task, the UnityGLTF \cite{unity_gltf} library was used to export models into the glTF (\textit{.gltf} and \textit{.glb}) \cite{gltf} format.
This library was primarly chosen as it could run without the Unity editor, but it had several other attractive properties besides that.

Firstly, both UnityGLTF and glTF should be stable options as they are maintained by the Khronos Group \cite{unity_gltf} \cite{gltf}, which is an open industry consortium responsible for open-source 3D graphics standards such as Vulkan, OpenGL, and WebGL \cite{khronos_about}.
Secondly, many 3D file formats are available, but glTF seems to be moving towards industry-standard judging by its extensive ecosystem and industry support \cite{gltf}.
The exported cities should thus integrate well with a wide array of industry software if glTF is used.
Lastly, the UnityGLTF library itself was easy to set up and configure.

Other 3D file formats were also considered but none of them had libraries that satisfied the requirements of this project.
Unity's FBX Exporter \cite{fbxexporter} could be used to produce \textit{.fbx} \cite{fbx} files, which is a proprietary but powerful format.
This library was especially attractive due to its ease of use and official support from Unity and Autodesk \cite{fbxexporter}.
Unfortunately, the FBX Exporter heavily depended on the Unity editor.
Wavefront's \textit{.obj} \cite{obj_files} and \textit{.mtl} \cite{mtl_files} files were also considered, but the libraries found either depended on the Unity editor or were deprecated.
Similar issues were also found with the COLLADA (\textit{.dae}) \cite{collada_files} format.
Hence, glTF using UnityGLTF was the only reasonable option found.

The correctness of the exported cities had to be verified somehow, and for this purpose the open-source 3D modeling software Blender \cite{blender} was used. 
The idea being that, if the models worked well in Blender, then in the worst-case scenario the cities could at least be re-exported from Blender to other tools.
Ideally, one would have constructed an automated pipeline and verified against multiple third-party software, but this did not appear necessary for this project.