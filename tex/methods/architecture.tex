\section{City Generation Architecture}
\label{sec:city-gen-arch}

In order to approach the problem of 3D city generation it was necessary to first break it down into manageable modules that could be worked on in parallel.
One of these modules had to be responsible for the GUI presented in the previous subchapter.
This module was named \textit{Application} and could be kept relatively small as most logic would be handled by the Unity engine.
There also needed to exist some module responsible for the PCG, since that is the core of this research.
This module was named the \textit{WorldGenerator}.
\textit{WorldGenerator} was then further split into eight submodules that would each be responsible for a distinct part of the model generation.

The implementation of the user interface would be rather simple and was thus treated as a single module, while the generation needed some architecture to handle the communication between submodules.
The two main architectures considered for the generation were a function-based and a pipeline-based approach (see Figure~\ref{fig:architecture_approaches}).

The pipeline had the advantage of less overhead since each generator would directly (or through an interface) pass data onto the next.
However, this lack of overhead could impose limitations and there still needed to exist some module responsible for receiving and exporting all the generated model data.
Furthermore, when Blomqvist et al. developed a PCG engine, they experienced issues with their pipeline architecture, such as interconnected modules and bias of workload towards the early pipeline stages \cite[p. 45]{ba_landscape}.
Thus, with these concerns in mind, it seemed reasonable to settle for the function-based architecture instead.

\begin{figure}[h!]
  \centering
  \begin{subfigure}[b]{0.48\textwidth}
    \includegraphics[width=\textwidth]{figure/architecture_functions.png}
    \caption{Function-based architecture. Each sub-generator is treated as an isolated function and \textit{WorldGenerator} controls how and when they are invoked.}
  \end{subfigure}
  \quad
  \begin{subfigure}[b]{0.48\textwidth}
    \includegraphics[width=\textwidth]{figure/architecture_pipeline.png}
    \caption{Pipeline-based architecture. The first generator is invoked by \textit{WorldGenerator}, and then each sub-generator passes on its output to the next sub-generator.}
  \end{subfigure}

  \caption{Two different architecture approaches considered for the generation logic.}
  \label{fig:architecture_approaches}
\end{figure}

The function-based architecture showed several promising advantages.
For one, each generator would only receive input that it actually depended on, while with a pipeline each generator would have had to pass along all its data.
This separation not only makes debugging easier, but it also improves performance slightly.
Another recognized advantage was that generators could now be run asynchronously and in separate threads, further benefiting performance.
Lastly, as generation steps would be invoked from GUI buttons, possibly in non-linear order and with \textit{undo} operations, it was better suited if \textit{WorldGenerator} could communicate with each sub-generator directly.
Consequently, the function-based architecture was chosen.

\begin{table}[H]
  \centering
  \begin{tabular}{lllll}
    \textbf{Input}                           &               & \textbf{Function}            &               & \textbf{Output}         \\
    \midrule
    \textit{Size, Offset, SeaLevel}          & $\rightarrow$ & \textbf{TerrainGenerator}    & $\rightarrow$ & \textit{Terrain}        \\
    \textit{Terrain, PopulationAmplifier[]}  & $\rightarrow$ & \textbf{PopulationGenerator} & $\rightarrow$ & \textit{PopulationMap}  \\
    \textit{Terrain, PopulationMap}          & $\rightarrow$ & \textbf{RoadGenerator}       & $\rightarrow$ & \textit{RoadNetwork}    \\
    \textit{RoadNetwork, PopulationMap}      & $\rightarrow$ & \textbf{BlockGenerator}      & $\rightarrow$ & \textit{Block[]}        \\
    \textit{Block, PopulationMap}            & $\rightarrow$ & \textbf{PlotGenerator}       & $\rightarrow$ & \textit{Plot[]}         \\
    \textit{Plot, Terrain, Population}       & $\rightarrow$ & \textbf{BuildingGenerator}   & $\rightarrow$ & \textit{Building}       \\
    \textit{Plot, Terrain}                   & $\rightarrow$ & \textbf{ParkGenerator}       & $\rightarrow$ & \textit{Park}           \\
    \textit{Plot, Terrain}                   & $\rightarrow$ & \textbf{ParkingGenerator}    & $\rightarrow$ & \textit{ParkingLot}     \\
    \bottomrule
  \end{tabular}

  \caption[]{The proposed generator functions needed to generate the 3D cities. The invocation of these functions are handled by \textit{WorldGenerator}. '\textit{[]}' denotes plural (e.g. list or array) of the preceeding data structure.}
  \label{table:generators}
\end{table}
\vspace{-0.4cm} % Mimic spacing below figures

An overview of the eight generators that were decided upon is shown in Table \ref{table:generators}, while the implementations and responsibilities of these generators will be detailed in the following subchapters.

\subsection{Terrain Generation}

Even though a city can be generated on flat space, such an implementation would have made the project seem too simplistic and dull.
A plane that could simulate a basic - yet somewhat realistic - environment for the city to be generated upon was from the beginning a self-evident feature that would be included in the project. 
The concept of this environment, which would eventually be named "terrain", was based upon common real world aesthetics and natural biomes, such as snowy mountains, grassland, hills and bodies of water (specifically oceans/lakes).
More aspects than that would bypass the range of the project’s goals.
One could spend an equal amount of time to create a beautiful terrain generator as to create the city itself, so a primitive terrain was sufficient.
This would in theory result in a sensible, nevertheless simple terrain such that more time could be spent on the main aim; the city itself.
The addition of the terrain feature would naturally require other generators to take into account the height of the terrain and avoid uninhabitable areas like water and very steep hills. 

These following tasks were the expected behaviour from running the terrain generator:
Create the base structure for the terrain
Generate the height levels of the terrain
Generate ocean/lakes
Texture the terrain based on parameters such as height levels and biomes

If there were enough time these 
Generate trees and shrubs (stretch goal)
Generate rivers that flow into lakes and ocean (stretch goal)

The terrain base could have been created in two ways: either using a mesh made of triangles or using the terrain API, an asset built-in into Unity.
Both options had their advantages and disadvantages.
Utilizing meshes would allow any shape of mesh to be created by connecting triangles to one another.
Further, one could also specify factors such as normals, colors, tangents, textures and so on.
However height-based texture splatting could be problematic.
On the other hand, the terrain API is easy to manipulate and encompasses extensive documentation.

In the first iteration for the terrain base, the terrain API by Unity seemed reasonable to choose because of simplicity. A tutorial by Brackeys guided the work in his tutorial about terrain generation, which greatly helped the early Unity stages.
It worked splendidly during the beginning, though eventually problems would be discovered.
The terrain API could unfortunately not easily be exported into the format wished for.
The terrain API had to be scrapped and replaced by terrain mesh instead, which was not a concerning change as visually it looked as good as before if not better. 

The next decision was to pick an algorithm that could designate height levels for the flat plane, i.e. to shape it with inward and outward facing dents that would represent valleys/ocean and mountains respectively.
Another self-evident aspect of the terrain was that if the city were to be procedurally generated, the terrain should be too, as it would allow drastically more varied content in the generator.
This is where noise was introduced.
The algorithms that could aid for this task was Simplex Noise and Perlin Noise.
These algorithms, when tied to points on the plane to determine its height, could shape the terrain in a manner similar to real world landscape.
The difference between the two algorithms are minor, although the decision ended up in favour of perlin noise for the reason that perlin noise already was included in one of the libraries of C#, which made implementing it much simpler and it didn’t require creating code or borrowing code from other programmers. 

The final aspect to include in the terrain was colour.
An empty, white canvas would not give the feeling of an appropriate location for a city.
Therefore textures were needed to grant the terrain the appearance for grass, mountains, water etc. to showcase their existence.
These following bullet points are what was expected of the textures:

Oceans and lakes should have a blue tint
Terrain slanting towards oceans with low steepness should transition into beaches
Spacious plains should cover various shades of green
Mountains should appear rocky
Tall mountain peaks should be covered in snow

Most of these were implemented, however beaches were omitted due to time constraints.

The textures were taken from Unity’s asset store community, along with the Standard Assets made by Unity themselves.
After a lot of searching, the most effortless choice for the water texture was taken from Standard Assets by Unity.
The rest of the textures were borrowed and resized from user ALP8310 on the Asset Store.

\subsection{Population Generation}

\begin{table}[H]
  \centering
  \begin{tabular}{lllll}
    \textbf{Input}                           &               & \textbf{Function}            &               & \textbf{Output}         \\
    \midrule
    \textit{Terrain, PopulationAmplifier[]}      & $\rightarrow$ & \textbf{PopulationGenerator}      & $\rightarrow$ & \textit{PopulationMap}        \\
    \bottomrule
  \end{tabular}

  \caption{Definition of the PopulationGenerator function which is responsible for generating intensity map based on the terrain.}
  \label{table:popgen}
\end{table}
\vspace{-0.4cm} % Mimic spacing below figures

% Short description
The population generator populates the terrain provided as input.
Specifically, a intensity map will be generated based on the terrain, which will represent the population density for the entire terrain.
Areas of high-density population can be altered with user input, which is given as a visual option during the road generation phase.
When placed, the population density of that area will be drastically increased.
This marker placement can be repeated multiple times to generate several cities.

PopulationGenerator is responsible for creating a procedurally generated intensity map describing the population in the world.
The terrain parameter is required to generate an intensity map since it needs to mask off certain areas in the landscape, such as oceans, rivers, or mountains.
Then, the generator would use a few layers of simplex noise to create a representation of populations throughout the world.
Population markers are used to increase population density in a certain area, and these are applied after generating the initial population map.
Markers are stored within the PopulationAmplifier set, which is the second input for the PopulationGenerator.
This is useful to make sure the created cities have a higher density, but it will still respect the original population map to some degree.

\begin{figure}[h!]
  \centering
  \includegraphics[width=0.5\textwidth]{figure/pop_density.png}
  \caption{An example of a population map with a Paris city generated within it.}
  \label{fig:pop_dens}
\end{figure}
\subsection{Road Generation}
The most basic aspect of a city that describes the overall city structure is the
road network. As such, it was important to get this right early on in the
development of the city generator. The application needed a method that was both
flexible and also offered realistic looking results.

At least two methods were considered when designing the road generator. One of
them were based on a recursive approach where the world is divided into multiple
cells and roads were placed between these cells. However, this type of algorithm
did not provide realistic looking results and, while being quite flexible, was
overly complex for creating a good road network.

Instead, the algorithm that was chosen was an Agent-based solution that works by
simulating road workers, which will be referred to as Agents, which only goal is
walking around the world depending on certain strategies and creating roads.
Agents could also decide to branch into multiple, new Agents, creating
intersections in the road network.
This algorithm resulted in realistic and good looking results and also offered
lots of flexibility. The goal of this chapter is describing how this method was
used to generate the cities in the application.

\subsubsection{Agent Strategies}
This section is dedicated to describing exactly what a strategy is. The use of
strategies means that Agents are considered to be ``dumb'', and can only move
around when it is told where to move. Which strategy to use depends entirely on
what the goal of the generation is.

Strategies also defines the configuration of the Agent, such as step size, how
many steps an Agent can take before terminating and how many times an Agent can
branch. The strategy that the Agent uses is responsible for deciding when an
Agent should terminate, except for step count which is handled automatically for
all strategies.

\subsubsection{Plotting Cities}
The algorithm requires some inputs in order to get started, and that consists of
a city type and a position. Then, depending on the city type, some other
variables are needed; Paris-like cities need a radius in order to define the
rough size of the resulting city, while Manhattan-like cities need a rough width
and height.

In the GUI of the application, the user needs to place down city markers in the
world with the desired position and size, then the generator can be started
which creates Agents that will act according to the city type that was chosen by
the user.

\subsubsection{Agent Road Generation}
The first step of the generation is to create a rough structure of the road
network. This includes the main roads of the city that is commonly found near
the center and boundaries of cities, but also highways outside the city.

The generation for main roads are different depending on the city strategy. For
example, Paris-like cities have distinct rings within the city boundaries, while
Manhattan-like cities generally follow a more organized and grid-like structure.
The generation of the main roads should mimic these characteristics.

So far, Paris and Manhattan-like cities have been described, but the algorithm
can be tuned to simulate other types of cities, the only thing that is needed is
a suitable strategy fo the Agents to follow. For example, if one wanted to
generate a chaotic strategy without any sensible rules, one could implement a
strategy that would make the Agents walk completely randomly.

Furthermore, Agents can switch strategy mid-generation, be it randomly or
depending on some variables that the Agent has access to. One example of this
might be that if an Agent detects that there is a relatively low population
density, it might decide to switch to an industrial type of strategy, which
could make multiple long, parallel lines of roads which is often found in
industrial areas in Sweden.

\subsubsection{Highway Generation}
The algorithm aims to create highways outside the boundaries of the cities,
which has a strategy that makes Agents prefer moving towards higher population
density. This type of strategy resulted in the tendency that connections between
cities are created.

Highways also have a lower probability of branching into more agents, which
resulted in roads that are longer with less exits, mimicking what can be seen in
the real world.

\subsection{Street Generation}
The street generator uses the same method as the road generator, which also
shows the level of flexibility of the Agent based approach. The only difference
is that these Agents start on any of the nodes in the road network and uses a
street strategy, instead of starting as a city type strategy.

Streets are generally the same for most cities, in the sense that they are
usually straight and follow a grid type of pattern. Slight variations does exist
in the real world, where some streets are curved rather than straight. Sweden
especially has streets that are more curved, while other countries such as
America generally are more grid-shaped. However, the basic street strategy that
the method uses mimics the grid-like street style.

\subsection{City Block Generation}

In this project, a city block is defined as a continuous area of land which is suitable in shape, size, and position for containing multiple buildings, parks, or parking lots.
The generation of such blocks is managed by BlockGenerator (see Table~\ref{table:blockgen}).

\begin{table}[H]
  \centering
  \begin{tabular}{lllll}
    \textbf{Input}                           &               & \textbf{Function}            &               & \textbf{Output}         \\
    \midrule
    \textit{RoadNetwork, PopulationMap}      & $\rightarrow$ & \textbf{BlockGenerator}      & $\rightarrow$ & \textit{Block[]}        \\
    \bottomrule
  \end{tabular}

  \caption{Definition of the BlockGenerator function which is responsible for generating city blocks.}
  \label{table:blockgen}
\end{table}
\vspace{-0.4cm} % Mimic spacing below figures

% The example labels are incorrect, but I could make a code PR that fixes that.
This function uses the road network from the road generation to find suitable areas of land to extract, and it uses the population map from the population generation to determine which type of city block it should label each area as.
Examples of such labels include: \textit{industrial}, \textit{suburbs}, \textit{downtown}, \textit{skyscrapers}, \textit{apartments} and \textit{parks}.

The labeling of city blocks was implemented in a rather simple manner.
Each label was restricted to only occur within a specific range of population density and was given a weighted probability of occurring compared to other valid labels.
In practice, this was accomplished by assigning each label a probability distribution that could then be queried using the population density of the city block area.
This approach makes it possible to distribute the frequency of labels based on population as well as proximity, which subsequently can be used to mimic patterns found in real-world cities such as San Francisco (see Figure \ref{fig:san_francisco}).

\begin{figure}[h!]
  \centering
  \includegraphics[width=0.6\textwidth]{figure/san_francisco.jpg}

  \caption{San Francisco skyline has a cluster of skyscrapers in the city center  and suburbs surrounding it \cite{san_francisco_img}. This pattern should be possible to mimic using weighted labels.}
  \label{fig:san_francisco}
\end{figure}

The trickier part was to find suitable areas of land to extract into city blocks.
The approach that was considered consisted of two iterations that both treated the road network as an undirected graph.
The first iteration would find all the minimum cycles in the road network graph and extract the area enclosed by each cycle.
These areas would then be treated as city blocks if their size and shape seemed suitable.
A limitation of this iteration was that all blocks would be perfectly enclosed by roads.

The intention of the second iteration was to supplement the first by producing new cycles instead.
The idea was to traverse the outskirts of the graph and attempt to extend imaginary nodes.
These nodes would then help form new cycles as seen in Figure~\ref{fig:extend_block}.
The imaginary nodes would solely be used for defining the areas of the new city blocks, and would not contribute to the original road network.
This iteration would help counter the limitation mentioned in the first iteration, but unfortunately, only the first iteration was implemented due to time constraints.

% Drawn using https://csacademy.com/app/graph_editor/
% Since I only show my creation (and not their site) I don't need to cite.
\begin{figure}[h!]
  \centering
  \includegraphics[width=0.3\textwidth]{figure/extend_block.png}

  \caption{Conceptual example of what the second city block extraction iteration could have produced. Here, the new cycle $BCDEF$ has been formed. Intersections are illustrated as nodes, roads as edges, and the extended block is colored green.}
  \label{fig:extend_block}
\end{figure}

The first iteration needed to find all minimum cycles in an undirected graph.
This is a known problem called Minimum Cycle Bases (MCB) and can be solved in $\mathcal{O}(m^3 + mn^2 \log n)$ or $\mathcal{O}(m^2n^2)$, for a graph with $n$ vertices and $m$ edges, using the efficient implementations proposed by Melhorn and Michail \cite{mcb_paper}.
These asymptotic time complexities might suffice for small cities, but not for multiple large ones.
Thus, either significant constraints had to be made, or another solution had to be found.

Fortunately, another solution was found.
Unlike typical graphs found in discrete mathematics, these road networks are constrained by a physical space, and are therefore a type of geometric graph.
Essentially, each node has a position relative to others and each edge has an angle relative to others.
With this information it actually becomes possible to solve the MCB problem, seemingly in only $\mathcal{O}(n + m)$.

The solution, first proposed by Petovan \cite{petovan}, can be described as follows:
\vspace{-0.5cm} % Reasonable spacing before list
\begin{enumerate}
  \item For each node, dispatch a \textit{turtle} along each connected edge.
  \item Let the \textit{turtle} traverse the graph by always following the edge closest (in counter-clockwise direction) to the previously traversed edge. Essentially, let the turtle always follow the rightmost edge relative to its current heading.
  \begin{enumerate}
    \item If the turtle tries to traverse some edge in a direction that has already been explored (potentially by another turtle), then terminate this turtle.
    \item If the turtle tries to traverse an edge it has already traversed, then terminate this turtle.
    \item If the turtle found back to the start node, then
      \begin{enumerate}
        \item if the turtle has turned an accumulated amount of 360 degrees counterclockwise, discard it.
        \item otherwise, add the nodes the turtle visited to a list of minimum cycles.
      \end{enumerate}
      %
  \end{enumerate}
  \item The resulting list of minimum cycles is the MCB of the graph.
\end{enumerate}

% Do we want to make a formal proof? Could be an appendix. Would be pretty cool to be first.
%The correctness nor runtime complexity of this algorithm has any known published proof, but an intution can still be gained.
Each turtle traverses the graph using the Pledge algorithm \cite{turtle_geometry}, which ensures that all non-terminated turtles will return to their start node.
The step \textit{2.a} ensures that all cycles have an area, and step \textit{2.c.i} handles the edge case where one turtle follows the perimeter of the whole graph.
The remaining cycles must be minimum since, otherwise, some edge would have to exist within the cycle but that edge would have been traversed by always following the rightmost edge, no matter the starting node (see Figure \ref{fig:graph_cycles}).
Thus, the correctness of the algorithm is shown.

% Drawn using https://csacademy.com/app/graph_editor/
\begin{figure}[h!]
  \centering
  \begin{subfigure}[b]{0.3\textwidth}
    \includegraphics[width=\textwidth]{figure/blockgen1.png}
    %\caption{A minimum cycle.}
  \end{subfigure}
  \begin{subfigure}[b]{0.3\textwidth}
    \includegraphics[width=\textwidth]{figure/blockgen2.png}
    %\caption{A non-minimum cycle.}
  \end{subfigure}

  \caption{Example of a minimum cycle (blue) and a non-minimum cycle (red). The red cycle has the edge \textit{AC} which splits the cycle into two minimum ones.}
  \label{fig:graph_cycles}
\end{figure}

Each edge is traversed twice (once from each direction) and each node is considered once for dispatching the turtles.
The rightmost edge can be queried in constant time by storing edges in counter-clockwise order in each node (and by storing array indices in the edges).
Thus, the algorithm runs in $\mathcal{O}(n + m)$.
With this efficiency, the generation of city blocks would no longer risk being a bottleneck.

Before returning, the BlockGenerator also insets the polygons of the resulting city blocks.
This step is needed to make sure the city blocks do not overlap with the road mesh of each edge.
Any further refinements of city blocks are subsequently handled by the plot generation.


\subsection{Plot Generation}  
A plot in this project was similarly defined with city block, where the difference is that a block can contain multiple plots. 
A plot can also have different labels with what kind of content that is generated within it. 
The label of the block, along with the population for each plot, is the deciding factor when plot labels is selected. 

\begin{table}[H]
    \centering
    \begin{tabular}{lllll}
      \textbf{Input}                           &               & \textbf{Function}            &               & \textbf{Output}         \\
      \midrule
      \textit{Block, PopulationMap}            & $\rightarrow$ & \textbf{PlotGenerator}       & $\rightarrow$ & \textit{Plot[]}         \\
      \bottomrule
    \end{tabular}

    \caption{Definition of the PlotGenerator function, which is responsible to split a block into one or more plot.}
    \label{table:plotgen}
  \end{table}
  \vspace{-0.4cm} 

PlotGenerator is needed because the blocks that are created often are too large to house, for example, a single building.
Instead of placing multiple buildings on a single block, using multiple plots leads to greater modularity to, for example, mix and match different plot labels on a single block.
Smaller plots mean less responsibility for the different content generators. 
They have the sole responsibility to fill their plot of land with whatever they like. 
Having different plots also grants the ability to have a large variety of plot labels within a single block. 
Something that can be quite complex if there is no clear separation.

Splitting a block into plots is essentially about dividing a polygon sensibly. 
Both block and plot are in the end a continuous area with defined points, the same as a polygon. 
Note that block size and population are not a part of splitting the polygons, but merely how the plot label is decided. 

One of the proposed solutions was to draw a line through the given polygon and let the new unique sub polygons be the result. 
This would have been a straight line going through the block. 
This does not, however, result in a good split necessary. 
It is hard to control the size of the polygons, as some can become quite small. 
There’s also the issue with concave polygons where it can become hard to create visually pleasing plots with consistency. 
There is also the issue where you cannot let the number of plots be a deciding factor to find the needed cut easily. 
This kind of algorithm would be outside of our scope of this project. 

% Add figures explaing.
An article about splitting polygon was found that explained how to split a polygon into N parts where each part would have equal size\cite{polygon_splitting_article}. 
In this article, you can find a more in depth explanation of the article. 
Each iteration finds possible cuts that could be made. 
The cuts are found by finding different ways to draw lines through what is left of the polygon. 
Some cuts are omitted in cases where concave polygons exist since they can be outside of the polygon. 
The cut that is closest to the size for each sub polygon is the one that is applied and added to the output. 
This was ultimately the algorithm used in the final product. 

As previously mentioned, block labels and population are factors when deciding the label for each plot. 
If the block label is, for example, a park or a parking lot, then the entire block is converted to one plot with the same corresponding plot label. 
However, if the block label is \textit{buildings}, then also the population with some randomness plays a big part in what label of the building is selected. 
If the population is high, then it is more likely that the plot label is a \textit{skyscraper} to accommodate the population. 

\subsection{Building Generation}
The purpose of the BuildingGenerator is to generate different kinds of visually pleasing buildings, everything from small houses to towering Manhattan-style skyscrapers. 
The BuildingGenerator has three inputs: the plot in which to build, the terrain on which to build on top of, and the population scalar that helps determine the size of the building.
The function spawns a building into the world. 

\begin{table}[H]
    \centering
    \begin{tabular}{lllll}
      \textbf{Input}                           &               & \textbf{Function}            &               & \textbf{Output}         \\
      \midrule
      \textit{Plot, Terrain, Population}       & $\rightarrow$ & \textbf{BuildingGenerator}   & $\rightarrow$ & \textit{Building}       \\
      \bottomrule
    \end{tabular}
 
    \caption{Definition of the BlockGenerator function, which is responsible for constructing a single building on top of a plot.}
    \label{table:buildinggen}
  \end{table}
  \vspace{-0.4cm} 

Seeing as buildings are one of the most important parts of generating a convincing modern city, it is essential to have a generator that can create different kinds of buildings. 
The problem of creating many different kinds of buildings was broken down into several different sub-generators, each responsible for a particular type of building.
The plot label helps decide what kind of sub-generator to use. 
The different sub-generators then have to take the inputs into account to generate a building.

% There are two different strategies implemented in this project for generating buildings: Using an L-system or importing an existing model of a building. 

% manhattan typ aktiga byggnader passar bra här
Multiple L-systems with a stochastic grammar were used to generate a unique set of buildings. 
The L-systems were implemented with type parameters to be able to handle the generation of any object, such as walls or floors.
The two types of parameters are the object type in the L-system and the data class that is used for the generation.
A building generated via L-systems has some generated floors, where each floor has walls and is built with multiple smaller textured wall segments. 
Examples of wall segments can be shop windows, normal windows, and doors.
The steps for generating a Manhattan-style skyscraper are as follows:
\begin{enumerate}
    \item An L-system is used to generate different kinds of floors for the building. Note that these floor types indicate what kind of L-system should be used later for the generation of the wall segments. Example of floor types can be:
    \begin{itemize}
        \item FirstFloor - A floor that has Shop Windows and Doors
        \item OnlyWindowFloor - A floor that will only have window segments. 
        \item MirrorFloor - Generate half of the floor segments and then copy and flip it for the next half of floor segments.
    \end{itemize}
    Images of these floor labels can be found in Figure \ref{fig:segmentsgen}.
    \item Each wall is then generated, floor-by-floor. Each floor type has its own L-system to generate different wall segments.
    \item After every wall has been generated; the building is put together inside the plot.
    \item A flat roof is generated at the top of the building.
    \item The building is then placed on the highest point of the terrain within the plot. Basement walls are then generated downwards to not make the building float in the air. %techically it's the highest point of the plot.points
\end{enumerate}

% Explain how the direction of the wall is decide. Maybe draw a picture of it?

% Add FirstFloor, OnlyWindowFloor and MirrorFloor images

\begin{figure}[H]
  \centering
  \begin{subfigure}[b]{0.32\textwidth}
    \includegraphics[width=\textwidth]{figure/FirstFloor.png}
    \caption{FirstFloor.}
  \end{subfigure}
  \quad
  \begin{subfigure}[b]{0.32\textwidth}
    \includegraphics[width=\textwidth]{figure/OnlyWindowFloor.png}
    \caption{OnlyWindowFloor.}
  \end{subfigure}
  \begin{subfigure}[b]{0.32\textwidth}
    \includegraphics[width=\textwidth]{figure/MirrorFloor.png}
    \caption{MirrorFloor.}
  \end{subfigure}

  \caption{Three different floor types and an example of their wall segment generation}
  \label{fig:segmentsgen}
\end{figure}

\subsection{Park Generation}
The ParkGenerator, much like the BuildingGenerator, has the primary task of filling city plots with visually pleasing, interesting content. 
To accomplish this, the ParkGenerator was implemented as a function that takes two inputs, the plot to generate the park in and the underlying terrain.
These parameters are then used to produce a park as output (see Table~\ref{table:parkgen}).

Since real-world parks come in all sizes and shapes, and can contain a variety of different objects, a scope had to be defined for what to include in the parks for this project.
For the scope of this project, the objects were limited to bushes, trees, rocks, and paths, but it was implemented in such a way that it would be simple to add more objects such as benches, fountains, and statues.

\begin{table}[H]
   \centering
   \begin{tabular}{lllll}
     \textbf{Input}                           &               & \textbf{Function}            &               & \textbf{Output}         \\
     \midrule
     \textit{Plot, Terrain}                   & $\rightarrow$ & \textbf{ParkGenerator}       & $\rightarrow$ & \textit{Park}           \\
     \bottomrule
   \end{tabular}

   \caption{Definition of the ParkGenerator function which is responsible for generating parks.}
   \label{table:parkgen}
 \end{table}
 \vspace{-0.4cm}
 
The development of the Park Generator consisted of two main phases:
\begin{easylist}
 @ Filling the park with objects such as trees, bushes, and rocks.
 @ Creating natural paths for people to walk on.
\end{easylist}
To make the parks vary in appearance through procedural generation, a way to randomize the placement of objects inside any given plot had to be constructed.
However, pure randomness of placement would produce some unrealistic results.
Another problem to tackle was the frequency of objects i.e. determining how much of each object is reasonable to have within each park. 
For example, one tree, 43 stones, and two bushes would likely look like an odd park.
 
The algorithm for generating objects was implemented by making use of uniform randomness between 0 and 10.
The range 0-10 was split into several separate, but not equally large sub-ranges, which would determine what object to generate.
For example numbers in the range [6,10] could result in a tree, [2,5] could result in a bush, and [0,1] could result in a rock.
As real-world objects are never identical, having one model for each object type would not be sufficient. 
Therefore, once the object type is decided, a model is sampled from an array of such model types, and then randomly rotated and scaled. 

It was considered important to be able to distinguish the park-objects based on size and type as these parameters would determine how far from each other objects would be allowed to be generated.
This is referred to as giving each object-type a radius relative to other object-types.
A function was then constructed to search within the radius of each object and make sure no object could be created within another object's radius (see Figure~\ref{fig:radius}).
This implementation is based on Poisson Disc Sampling \cite{poisson_fast} in a manner that it scatters points randomly across the plot, while maintaining a distance from each already created point.
Here the object-radius is used in order to determine how closely objects may stack. 
When the paths were implemented, a slight adjustment also had to be made to the algorithm for object placement, such that all objects took into consideration the coordinates of the path and were not placed on top of it.
\begin{figure}[H]
\includegraphics[width=\linewidth]{figure/radiuscontrol.png}
\caption{Radius of the tree object shown in pink. This radius indicates how closely trees may pack.}
\label{fig:radius}
\end{figure}

The algorithm for creating paths was inspired by Cyclic Dungeon Generation \cite{cyclic}.
The main takeaway from that concept was that a start and goal node was generated for the algorithm to create a loop between.
Having a loop would allow for people to take a stroll around the park and return to where they entered, and based on observation, this is also a frequently occurring pattern for paths in real-world parks. 

The development of the path generation can be described as three iterations of implementation.
The first implementation of the algorithm created two random points within the plot, one as start and another one as goal. 
The program then traveled the park at random until it found a path to the goal, at which point it would find a new way back to the starting point. 
It was however, quickly discovered that this approach had some problems.
 
One of these problems were that the paths would look unnatural when taking steps far away from the goal. 
A solution to this was to revert the path to its previous node if it took a step which was further away from the goal than it had previously been, this was an instant improvement in appearance. 
However, after observing many of the generated paths from this approach the paths still looked linear. 
This was believed to be because simply going from one point to another and not straying off too much from the goal could not produce much more interesting results than that (see Figure~\ref{fig:linear}).
\begin{figure}[H]
\includegraphics[width=\linewidth]{figure/linear.png}
\caption{Example of a path generated by the first implementation of the algorithm.}
\label{fig:linear}
\end{figure}
The second implementation of the algorithm was inspired by the flaws which had been demonstrated by its predecessor, more specifically its linear shapes and its tendency to create steps with abrupt angle-changes.
As a result of this, it was for the second implementation decided that rather than having a start and goal, the path would consist of an array of points, which all had to be visited before the path would be completed.
This was changed in order to try to get rid off the linearity presented in the first implementation.
For this implementation a control of how many degrees a step in the path could differ from the current point to its previous step was also added.
This was done to avoid the zigzag-like patterns caused by the angle-changes from the first implementation.
The second implementation would just like the first one look for one assigned point, but in this case start searching for the next point after finding the previous one and continue doing so until all points have been found (see Figure~\ref{fig:texsplat}). 
\begin{figure}[H]
\includegraphics[width=\linewidth]{figure/texturesplat}
\caption{Example of a path generated by the second implementation of the algorithm.}
\label{fig:texsplat}
\end{figure}
The third and final implementation made use of a similar algorithm.
Some fairly small changes were made in order to address the issues experienced with the second implementation.
The second implementation would still create paths starting in the middle of the park which did not make a lot of sense, this was handled by always creating the start node on one of the edges of the plot.
The algorithm then creates an additional number of randomly placed points, based on the size of the plot.
Having this number be based on the size of the plot made for some more interesting and realistic results which was lacking in the second implementation.

Afterwards, the points are sorted by distance to enforce that the algorithm always traverses towards the point closest to itself. 
This change was inspired by seeing some paths traverse towards points far away from the starting point only to then traverse to a point close to the starting point.
Once all goal points have been reached a path is either created back to the starting node, or a new path is created to a randomly selected edge on the plot. 
This change was made to stop enforcing loops and to add more variety to the generated paths.
Finally, the algorithm creates exits/entries to the park by projecting a line to closest edge of the plot from some of the points, and then creating a path between them. 
 
The visual implementation of the paths was done by texture splatting on top of a Unity Terrain for the first two implementations (see Figure~\ref{fig:linear} and Figure~\ref{fig:texsplat}).
However, as the Unity Terrain needed to be replaced with a mesh generated one, this had to be re-implemented for the third implementation.
Some different approaches for visualizing the paths were investigated, such as using quad and sphere meshes and then simply placing these along each path.
The spheres would cause a shape that was still distinguishably circular and the quads would likewise create patterns that were clearly square. 
The final implementation would instead make use of the logic from the RoadGenerator to create its own internal road network, thereby inheriting the functionality used to generate the corresponding mesh (see Figure~\ref{fig:visuals}).\begin{figure}[H]
  \centering
  \begin{subfigure}[b]{0.4\textwidth}
    \frame{\includegraphics[width=\textwidth]{figure/splinepath}}
    \caption{Visualization of a path using splines.}
  \end{subfigure}
  \quad
  \begin{subfigure}[b]{0.3\textwidth}
    \frame{\includegraphics[width=\textwidth]{figure/spherepath}}
    \caption{Visualization of a path using spheres.}
  \end{subfigure}
  \caption{Two of the approaches for visualizing paths on top of the mesh generated terrain.}
  \label{fig:visuals}
\end{figure}

\subsection{Parking Lot Generation}
The ParkingGenerator is responsible for filling roughly ten percent of the generated world's plots with parking lots.
This is based on research conducted in the city of Phoenix, Arizona \cite{percent}.
The ParkingGenerator is implemented as a function that takes a plot and its underlying terrain as input and produces a parking lot as output (see Table~\ref{table:parking}.
\begin{table}[H]
   \centering
   \begin{tabular}{lllll}
     \textbf{Input}                           &               & \textbf{Function}            &               & \textbf{Output}         \\
     \midrule
     \textit{Plot, Terrain}                   & $\rightarrow$ & \textbf{ParkingGenerator}       & $\rightarrow$ & \textit{Parking lot}           \\
     \bottomrule
   \end{tabular}

   \caption{Definition of the ParkingGenerator function which is responsible for generating parking lots.}
   \label{table:parking}
 \end{table}
 \vspace{-0.4cm}


One approach for generating parking lots for could have been using Esri CityEngine's \cite{Esri} one. % having a hard time finding this source, Anton who found it... plz help me. 
What they use for some of their generation is simply filling areas with textures, this is used for their parks as well as their parking lots. 
So for an example, their plots designated to be parking lots would simply be painted over with a \textit{parking-lot} texture. 
This was at first considered to be a viable option for the project, only it offers little alternatives for modification. 
Modification including shapes of the full parking lot as well as the size of the individual parking spaces. 
As every other generator provides content that is fully scalable it would be inconvenient to make use of a texture that could not be rescaled as easily, if at all.

The approach instead elected for was an algorithm that generates the parking spaces individually.
This algorithm works by first using a function for approximating the largest rectangle inside any given polygon. 
The reasoning behind this was the group's observation that a lot of parking lots seem to be shaped in rectangular ways (see Figure~\ref{fig:parkings}).
\begin{figure}[H]
  \centering
  \begin{subfigure}[b]{0.56\textwidth}
    \frame{\includegraphics[width=\textwidth]{figure/parking1}}
  \end{subfigure}
  \quad
  \begin{subfigure}[b]{0.395\textwidth}
    \frame{\includegraphics[width=\textwidth]{figure/parking2}}
  \end{subfigure}

  \caption{Two examples of parking lots observed by the project group, showcasing the rectangular shapes mentioned above.}
  \label{fig:parkings}
\end{figure}
Afterwards, based on the size of the rectangle, the algorithm would then generate either two, or four columns of parking spaces (see Figure~\ref{sizebased}.
