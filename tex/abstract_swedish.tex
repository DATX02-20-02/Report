\chapter*{Sammandrag}

Denna rapport undersöker hur metoder inom \textit{Procedural Content Generation} (PCG) kan kombineras med datorgrafikteori för att procedurellt generera virtuella 3D städer, lämpade för användning inom datorspel, film, och reklam.
Denna undersökning utfördes med syftet att underlätta den annars tidskrävande processen av att modellera sådana städer manuellt.
Utredningen studerar tidigare arbeten och existerande PCG metoder för att föreslå ett nytt system kapabel till att generera moderna städer från grunden.
Potentialen av detta system demonstrerades sedan via utvecklingen av en stadgenerationsprogramvara.

Som ett resultat av denna undersökning, så bidrar vi med \textit{CityCraft} -- ett kostnadsfritt, \textit{open-source}, MIT-licensierat datorprogram som interaktivt genererar moderna städer i realtid.
De genererade städerna kan exporteras som 3D modeller till det avgiftsfria filformatet glTF, vilket gör de kompatibla med flertalet tredje parts verktyg där de kan förfinas vidare.
Kvaliteten på dessa städer åstadkommer inte den kvalité som manuellt tillverkade städer tenderar att ha.
Däremot tar generationen endast en bråkdel av tiden som manuell modellering kräver.
Dessutom kräver applikationen inga förkunskaper inom modellering eller programmering för att kunna användas effektivt.

Realiseringen av CityCraft påvisar potentialen av att kombinera PCG och datorgrafik för att automatisera processen av att modellera städer.
Det bidrar även med insikt till huruvida specifika metoder, så som Agent-baserad väggenerering, L--system-baserade byggnader, och \textit{Level of Detail} (LOD), kan integreras för att uppnå goda resultat.
Ambitionen med detta bidrag är att CityCraft kan agera som en språngbräda till fortsatt forskning inom stadsgenerering inom \textit{open-source} domänen.