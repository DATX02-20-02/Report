\subsection{Tools}
These are a few of the tools that we have selected to use to create our project.

\subsubsection{Git and Github}
Git was chosen as it is the industry standard for version control of code.
For collaboration purposes the git repository will be hosted on GitHub.
Most project members had previous experience with both of these technologies,
and that ended up becoming the deciding factor to use these technologies as opposed to similar technologies such as Subversion and Gitlab.

\subsubsection{Unity and C\#}
The project needs a GUI for the visualization of cities, so there is a need for some type of library, framework, or engine that can eleviate some of this work.
LWJGL, JMonkeyEngine, Unity, Unreal, and Godot were all options that were considered in this project.
Unreal was excluded because of its hard learning curve, and the way its favors visual
programming using something called \textit{blueprints}.
Godot was excluded for its limitations in 3D and for its young and small community.
LWJGL and JMonkeyEngine were disregarded for the amount of work that would have been required to get simple stuff like roads andvisual debugging to work.
The choice ended up being Unity as it will allow us to focus on problem of PCG at hand.
Further arguments for using Unity were its large community, extensive range of tutorials, good documentation, stability, cross-platform support, and its wide array of first-, and third-party tools.

Unity officially only supports C\#, and since this language is similar to Java (a language all project members had previous experience with) the choice of programming language felt obvious.