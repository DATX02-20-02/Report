\subsection{Terrain Generation}
We will approach this subproblem by first generating a height map that represents the hills and the valleys of the landscape. 
We plan to generate the height map using multiple layers of simplex noise.
Simplex noise is a type of gradient noise commonly used in computer graphics. 
Gradient noise is one out of two commonly used noises used for terrain generation, the other being value noise.  
The concept of value noise is creating lattice points with pseudorandom values. 
The noise function then returns the interpolated values of the surrounding lattice points. 
Where as gradient noise creates a lattice of pseudorandom gradients, the dot products of the gradients are then interpolated to obtain values between the lattices. 
The user will provide a value for the sea level in the form of a number, and we will color the terrain based on the height levels e.g.\ grasslands near sea level, rocky mountain textures for the taller regions, and snow for top peaks of some of the taller mountains.
The landscape might end up looking something like this, although with less water and mountains, and more flat areas for cities to be placed.

If there is still time after previously mentioned implementations have been made we intend to make the terrain more interesting by generating trees and grass throughout the world.
The way we intend to do this is to use poisson disc sampling to distribute points for trees, and then for grass. % source: https://odr.chalmers.se/handle/20.500.12380/244588
We should sample trees first since they will have a larger radius.  