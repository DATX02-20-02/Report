\subsection{Terrain Generator}
\begin{center}
  \textit{User parameters} $\rightarrow$ \textbf{TerrainGenerator}  $\rightarrow$ \textit{Terrain}
\end{center}

This subproblem will be approached by first generating a heightmap whose pixels values will represent the terrain elevations, which will then be rendered as a 3D mesh.
One limitation of using heightmaps is that they can not represent more complex 3D geometry such as caves or overhangs.
However since the main focus of the project is on procedurally generating cities, and not terrain, this is not regarded as a problem.

We plan to generate the heightmap using multiple layers of simplex noise.
Each octave, or layer, of simplex noise will represent different levels of terrain granularity.
Higher frequency of noise corresponds to an area on the heightmap where we would like to generate an area with a small, yet intense difference in elevation.

Whereas lower frequencies will correspond to areas of a less intense, but larger difference in elevation forming the contour of the terrain.
This can be used to generate both smooth hills and rocky mountains.

The color of the terrain will be based on the height levels e.g.\ grasslands near sea level, rocky mountain textures for the taller regions, and snow for top peaks of some of the taller mountains.
One approach for achieving this is a technique called Texture splatting that is commonly used for terrain rendering.
This method consists of combining different textures blended according to an alphamap, also referred to as a weightmap or splatmap.
Figure~\ref{fig:texture-splatting} demonstrates this technique in action.

\begin{figure}[H]
  \centering
  \includegraphics[width=0.4\textwidth]{figure/texture-splatting.png}
  \caption{Example of texture splatting from related work \cite{wiki:texture-splatting-img}}
  \label{fig:texture-splatting}
\end{figure}

Furthermore, the user can provide a threshold height for the sea level where any point of the terrain below this threshold value will be filled with ocean. 
This can simply be done by drawing an animated water texture on a plane positioned at said height level. 

As previously mentioned in the problem section, some things could be added to our project in the later stages, if there is time for them. 
We could, for example, add water flowing from mountains that turn into rivers, and similarly, we could make these rivers flow into lakes to make our terrain look more natural.
For now, we have only speculated on how we would like to approach this problem, but we have discussed forming rivers using a gradient descent starting from the mountain-tops.
The way a river flows would be determined by \textit{river agents} based on the gradient descent function from mountain-top to ground level. 
We could then decide to, depending on where the river agent \textit{terminates}  to perform different actions on that part of the terrain.
If a river flows into the ocean we would not need to create anything, but if a river flows into seemingly nothing we could decide to generate a lake there. 

Another approach would be to generate the lakes seemingly arbitrary at first, and then try to find a gradient descent path from the mountain-tops to the lakes.
If no such paths are found, we do nothing, but if the agents finds a path we generate a river from the mountain to the lake. 

Other candidates for terrain generation include Search-base, ML, and Voronoi diagrams.
The first option was excluded because of the difficulty of defining what a ``good'' terrain is using concrete rules.
ML could be used with real terrain data, but such a solution may be unnecessarily complex given that terrain is not the focus of the problem.
Voronoi diagrams could be used to divide regions of the terrain, but a previous BA project had several difficulties and setbacks with them \cite{ba_landscape}.
Therefore, we have chosen to not pursue this approach.
