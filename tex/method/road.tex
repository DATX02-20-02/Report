\subsection{Road}
\begin{center}
  \textit{Terrain, Population map} $\rightarrow$ \textbf{RoadGenerator} $\rightarrow$ \textit{Road network} 
\end{center}
Road generation is what will design the entire city, and it will do it by placing different kinds of roads.
We will approach this generation by first allowing the user to decide where city generation shall commence by giving them the option to place a population markers for the roads to be constructed.
The size of the city generated is defined by the user, maximum being the size of the available landscape.
Cities far away from each other will be connected by highways, and there can exist a randomly generated 

\subsubsection{Plotting out cities}
The user should be able to specify exactly where cities should begin their generation sequence.
They will do this via population markers.
Each city has a type which describes the general appearance and layout of the city.
Some examples of city type generation strategies are;
\begin{itemize}
  \item Paris strategy, which would generate circular cities with roads extending from the middle and outwards.
  \item Manhattan strategy, which consists of lots of straight roads going through the city perpendicular to other roads, creating a grid-like appearance.
  \item Chaotic strategy, completely random with lots of turning, no structure whatsoever.
\end{itemize}

Multiple cities can be plotted out at once, and generally when a road leaves the bounds of the city it converts to a highway which aims to connect to other denser areas.

\subsubsection{Main road generation}
Main roads, and roads in general, are generated by what is called \textit{Agents}.
These only have one job, and that is to walk arond the landscape and leave roads behind them.
The agents generate the main roads depending on if it is inside a city, and what type of city it is.
All agents have a strategy that they will follow, instructing them how they should move around.
Agent strategies also tells the agent when it should terminate.

This type of generation results in a lot of flexibility, because each agent can be instructed to walk around differently.
For Paris strategy, some agents are configured to walk around in circles around the center point and others are configured to walk roughly straight outwards from the center.
For Manhattan strategy, the agents simply create completely straight roads, some agents are rotated by 90 degrees to create the perpendicular roads.

This also means agents can decide to switch strategy depending on some criteria, for example when an agent leaves the bounds of a city, it can change its strategy to a highway-type strategy.

Highways are a type of main road, and agents that have a strategy that creates highways should attempt to follow the population map towards higher density areas.
This helps cities to connect to eachother, because highways generally do not turn around as much as other main roads and it will eventually find a higher density area in the population map to go towards.

However, higher density areas does not mean there is always a city located there, it might find an area without a city.
The idea is that the highway strategy can also create new cities or villages if it comes across a very dense area and has not connected to an existing city.
There are cases where a city can be created right next to another city, but that would just result in both cities merging into one.

\subsubsection{Street generation}
While generating main roads, new agents are created which branch off the main road to create streets.
These are generally straight and grid-like, to mimic the patterns found in most neighborhoods.
However, different kind of street strategies can be created to mimic other types of neighborhoods.
To ensure that streets do not cut off main roads, these agents are given a lower priority which ensures that they are always created after main roads are finished generating.
The agents are then prioritized according to the order they were created, which means an agent will generate an entire neighborhood before the next street agent begins.
This creates clearer patterns around the city with less agents competing over the same area.

Agents with a street strategy will always look up the population map before placing a road or branching.
Higher density means a higher chance of branching, creating denser neighborhoods near the center of the city.
An agent that tries to travel into a place with too little of a population density will simply terminate without creating a road. 
