\subsection{Terrain Generation}

The problem of terrain generation can be further divided into the following tasks.
\begin{easylist}
  @ Generate the heights of the terrain.
  @ Texture the terrain based on height levels.
  @ Generate trees and shrubs.
  @ Generate ocean.
  @ Generate lakes.
  @ Generate rivers that flow into lakes and the ocean. (optional)
\end{easylist}

The terrain heights need to represent the hills and valleys that define the silhouette of the world.
The textures should aid in visually describing the landscape, as well as
help contribute to an overall more realistic model. The following list presents
a few concrete examples of such texturing.
\begin{easylist}
  @ Oceans and lakes should have a blue tint.
  @ Landscape slanting towards oceans should transition into beaches.
  @ Spacious plains should cover various shades of green.
  @ Mountains should appear rocky.
  @ Tall mountain peaks should be covered in snow.
\end{easylist}

In the context of realism, the generation should assume a season somewhere
between spring and summer.
This scope could potentially be extended to include other seasons as well as
different types of landscape types.
However this is only intended as an optional aspect that will handled in the case
of excess time.

To make the terrain more convincing there will also be generation of trees and shrubs
throughout the world. These should grow in groups, just like forests in
real life, and they should not intersect with any roads or buildings.

The generated lakes and oceans will both restrict where roads and buildings can be placed.
The same goes for rivers if that strech goal is met.
These aspects should help create more interesting interactions between terrain and city.

The overall terrain should follow a realistic aesthetic. Consequently it should not
attempt follow a \textit{Low Poly} aesthetic nor be made up of visible \textit{Voxels}.