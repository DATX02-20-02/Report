\subsection{Terrain Generation}

The problem of terrain generation can be further divided into the following tasks.
\begin{easylist}
  @ Generate the height levels of the terrain.
  @ Texture the terrain based on parameters such as height levels.
  @ Generate trees and shrubs.
  @ Generate ocean and lakes.
  @ Generate rivers that flow into lakes and the ocean. (stretch goal)
\end{easylist}

The terrain heights need to represent the hills and valleys that define the silhouette of the world.
The textures should aid in visually describing the terrain, as well as help contribute to an overall more realistic model.
The following list presents a few concrete examples of such texturing.
\begin{easylist}
  @ Oceans and lakes should have a blue tint.
  @ Terrain slanting towards oceans with low steepness should transition into beaches.
  @ Spacious plains should cover various shades of green.
  @ Mountains should appear rocky.
  @ Tall mountain peaks should be covered in snow.
\end{easylist}

In the context of realism, the generation should assume a season somewhere between spring and summer.
This scope could potentially be extended to include other seasons as well as different types of terrain types.
However, this not intended to be a part of this project's scope as the potential improvements are not deemed worthwhile.

To make the terrain more convincing there should also be a generation of trees and shrubs throughout the world.
These should grow in groups, just like forests in real life, and should not intersect with any roads or buildings.
The placement of trees should look natural in the sense that they grow in a group, but also do not grow to tightly packed.
Trees in real life need a lot of soil, and the generation should respect that.

The generated lakes and oceans will both restrict where roads and buildings can be placed.
The same goes for rivers if that stretch goal is met.
These aspects should help create more interesting interactions between terrain and city infrastructure.

The overall terrain should follow a realistic aesthetic.
Consequently, it should not replicate a \textit{Low Poly} aesthetic nor be made up of visible \textit{Voxels}.
The same goes for roads and buildings.