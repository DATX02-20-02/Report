\subsection{Building Generation}

The problem of building generation can be further divided into the following tasks.

\begin{easylist}
  @ Generate different types of buildings depending on population.
  @ Generate villas.
  @ Generate terraced houses.
  @ Generate apartments.
  @ Generate skyscrapers.
  @ Generate parks.
  @ Generate stores, restaurants, hotels, and more. (stretch goal)
\end{easylist}

The relation between population and building types does not be realistic in
terms of scale, but there should be a clear positive correlation.
Areas with metropolitan-levels of population should have a high frequency of skyscrapers, apartments, and other tall buildings.
Medium-density areas should contain apartments and terraced houses, while low
density areas should mostly contains villas.

Other buildings such as stores and restaurants would help enrich the cities with
more variety, but they might require a lot of time spent on modeling.
Because of this they have been defined as stretch goals.
However there are no restrictions on the appearance of any buildings (even the mandatory ones) as long as they have clear real-life equivalents.

Although parks are not technically buildings, they will be included here for the sake of brevity and because of their similarity.
Like buildings, parks will vary depending on population density.
In high-density areas they should occur relatively frequently, especially large ones, but the they should slowly be replaced by areas of nature as one approaches the outskirts.