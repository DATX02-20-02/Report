\subsection{Road Generation}

The problem of road generation can be further divided into the following tasks.

\begin{easylist}
  @ Generate clusters of main roads that represent cities where the population is high.
  @ Generate streets between the main roads.
  @ Connect roads with intersections or roundabouts.
  @ Ensure plots of land between roads where buildings can be built.
  @ Ensure that the road network forms a single connected graph for each landmass.
  @ Generate sidewalks along roads.
  @ Generate bridges over bodies of water to ensure a single connected graph. (stretch goal)
\end{easylist}

All roads should be placed on the ground and may not cross any bodies of water, except for bridges.
The roads also need to adjust themselves based on the elevation levels of the terrain.
This means that the roads need to be able to form smooth curvatures as they traverse slopes.
Both smooth curvatures and sharp corners need to be possible when the roads turn to the left or right of their orientation.

The road generation may modify the height levels of the terrain to form smooth surfaces, but it may not do so if the result causes unnatural elevations.
As an example, connecting two tall hills by raising all the land in-between is not considered natural.
However, flattening a slightly bumpy surface into a smooth one is considered natural.
As a guideline, no level heights should be modified beyond the real-life equivalent of 1 meter, unless it can be motivated by real-life examples.

There is also a restriction on how steep slopes the roads are allowed to directly traverse.
A road may not reach inclines that can not be supported with common real-life equivalents.
Although roads may not directly go up too steep hills, they can still form zig-zag patterns or spirals to reach the top of most peeks.

Finally, roads must not overlap with each other, and the streets may not create unrealistically small areas for buildings to be built on.