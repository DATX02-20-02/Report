\section{Future Work}
This subchapter highlights the possibilities of expanding CityCraft and the system it implements, mainly focusing on different ways to evolve the application beyond what we had time for.
An obvious way to further develop CityCraft would be to simply add more aspects to the generators, such as additional road network strategies, buildings types, and objects inside parks.
This would not only be a good way to stress the system, but would also make the generated cities further resemble reality.
 
As the terrain was never a core focus of CityCraft, it would be fitting to give users the option of providing their own terrain models, which the cities would adjust to.
Users could also be given the ability to draw their own population map, as well as the ability to draw new roads during runtime.
Furthermore, content such as road markings, road signs, and possibly bridges could also be added.
  
One way to further develop the building generation would be to simply include a larger set of textures, or to algorithmically generate textures in order to further increase visual variety.
Another area of interest is to generate the interiors of buildings as well as adding public transport infrastructure.
As for the parks, the generation of paths was implemented in a way that did not consider concave polygons.
This could potentially be solved by either altering the algorithm for paths, or altering the way that plots are divided.

However, the primary goal of future work should be to validate the theoretical system that CityCraft implements.
This could be done by comparing the algorithms and architecture proposed in this report with other approaches.
For instance, Voronoi Diagrams, WaveFunctionCollapse, and Machine Learning are all potential options for such comparisons.
By doing this, one could better identify the possible strengths and weaknesses of our proposed system.
It will also be important to validate if the application GUI meets the user-friendly requirements by conducting user tests.