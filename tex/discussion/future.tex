\section{Future Work}
The areas which was concluded by the group to be of interest when conducting future work was slightly mentioned in the results chapter, however here they are discussed more in depth.

The generation of the terrain could be further developed to include such features as rivers flowing from mountain tops and also having more climates than the ones we stuck with.
This could include such things as massive forests, saharan-like deserts or snow-covered landscapes.

The generation of the road network could in practice be worked on infinitely.
By constantly developing new strategies and refining the way the agents work, one could end up generating strategies for each city known to man.
However, a more sensible way to address further working on the road network would be to simply add a couple more strategies, after investigating what more common road networks look like in the real world.

Generating buildings is also a task of seemingly endless variety. 
The future work most likely to be adapted here would be to include a larger set of textures, or algorithmically generate textures in order to further increase the variety of building's appearance while not limiting them to the placement of said textures. 

The park generation could mainly benefit from adding more objects, and finding suitable ways to fit them in the world.
Some such objects were touched on in the method chapter for this generator, but a lot more could be added.
Changes could also be done in size, as well as shape in order to achieve more varying looks. 
One could give some cities each a probability of generating something similar to Hyde Park (London) or Central Park (NYC) which would likely produce some very interesting results.

The parking lot generation could simply be extended by adding more shapes than the rectangular ones used in this project.


