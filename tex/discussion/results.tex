\section{Quality of Results}
\begin{easylist}
  @ \textbf{Question}: Do models correctly integrate with third-party software such as Blender \cite{blender}? \newline
  	\textbf{Answer}: Yes the models integrate well with Blender, which is the only model-software we've tried importing our models with. % tror jag?
	 
  @ \textbf{Question}: How well is the codebase structured for replacement and expansion of features? \newline
  	\textbf{Answer}: As the generation is divided into a number of sub-generators it would be easy to add more generators on to the total generation.
	Furthermore the generators themselves are structured in a way that adding more content to them, or modifying them should be a simple adaptation to make. 
	
  @ \textbf{Question}: How much notable variety is there in the generated content? \newline
  	\textbf{Answer}: As this might be the hardest question to answer simply by text, here are two resulting cities from the application with varying size and shape.  % Löser bilder på detta senare tänker jag
	% Ha mer beskrivning kring figuren om skillnaderna snarar än här. 
	
  @ \textbf{Question}: How much control do the users have over the generation? \newline
  	\textbf{Answer}: The user has a lot of control over the generation however some areas of this could still be improved. 
	Marking the territory on which the user wants to place it city is not entirely accurate when making the marked area very small. 
	On the bright side the user has the ability to undo the work of a generator if that has left him unpleased, an area of improvement here would have been to allow the user to mark a territory of the world after generation is done and 	be able to re-generate that section. This however was deemed quite time-demanding and will be left as an area of future work. 
	
  @ \textbf{Question}: Are there any known visual artifacts in the models? \newline
  	\textbf{Answer}:  One visual artifact we are aware of would be that buildings as well as parking lots are generated right on top of the terrain.
	The larger flaw resulting from this would be that it would seem impossible for theoretical cars to reach the generated parking lots, without driving on top of the grass that is.
	Had there been more time, a solution to this would have been implemented by the group but a lot of things were prioritized more heavily and this is therefore also left as an area of future work. 
	two of the approaches that were considered were either filling the entire plots with asphalt (which would probably have looked the best had there been a suiting texture) and would definitely have looked the best for the buildings.
	The other approach that was considered was somehow creating road from the road network to the parking lot, this however was deemed to be very difficult due to the design of the road network. 
	% Resten lämnas som övning till Anton Håkansson. 
	
  @ \textbf{Question}: Are there any known bugs or crashes in the application? \newline
  	\textbf{Answer}: No. % Hoppas detta stämmer vid tiden för inlämning, men we'll see :D 
	
\end{easylist}
