\section{Process and Workflow}

One main takeaway from this project was that even though we believed our defined scope was reasonable, we didn't take into account how much time one could spend attempting to perfect the parts that were already working.
An area where this became particularly clear was performance.
During the course of the project we would often find the generators creating content which was visually sufficient for what we had hoped to achieve, but the process of generating would be slow. 
The three main parts of the project where we observed that we could increase performance were the following:

\begin{easylist}
  @ Decreasing the resolution of which we import our textures.
  @ Decreasing the triangle-count of buildings.
  @ Using Level of Detail~(LOD).
  @ Combining meshes. 
 \end{easylist}
 
The first part of the list was simple to tackle and did not require any additional code to be written, we simply had to alter the resolution at which our textures were imported.
This came from us noticing that the textures were being imported at a default around 2000 pixels, which we could decrease to a staggering 64 and still reach almost indistinguishable results visually, at a dramatic cost in efficiency. 

The second problem was revealed by the fact that after generating buildings FPS suddenly became a major concern. 
After inspecting this issue we found that the buildings were being composed of a lot more triangles then they needed and this was fixed by ...

% maybe move LOD to road-mesh methods
The third improvement of performance was using a technique in computer graphics known as level of detail~(LOD).
LOD typically works by decreasing the visual quality of an object based on some function of the camera distance to that object.
Typically, different distance intervals are configured to corresponding to different quality meshes.
The further away the camera is the lower the quality of the mesh becomes as to improve the efficiency of rendering without a noticeable difference to the user.
In CityCraft, this technique was used extensively for the roads, intersections and buildings and resulted in significantly improved the performance.

The fourth and final major adjustment made for increasing performance was to combine some of our generated meshes. 
Combining meshes for the building generator, when using the L-system strategy, was crucial for lowering memory usage. 
The first version had a mesh for each wall segment, and each wall segment had its material with an accompanying texture. 
The consequence was memory usage peaking over 10GB with a medium-large city. 
There was also the issue with each wall segment was their object spawned into the world.
This object, called GameObject in Unity, has its render, which also meant an increase in the number of batches needed.
Combining the wall segments into one building meant a drastic change in memory usage. 
Only one of each unique material that represented the different wall segments was loaded in Unity.   

% THEO HELP 

Something that could have been utilized more efficiently had we perhaps chosen a more collaborative style when developing our generators rather than dividing them among us, would be re-using already implemented code. 
This became blatantly obvious when working on the path for the parks.
At one part of the project, code was being written for generating meshes for the park paths, while code for doing this had already been written and was being used in the road generation. 
Another problem that arose from this was that the parking lot generator would have probably had to be developed more in collaboration with the road generator in order to find a way to ensure that roads always have an entry to the parking lots.

