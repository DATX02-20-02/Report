\section{Method}
The subproblems in section \ref{sec:problem} have been divided further into the generators found in figure \ref{fig:generators}. 
The basic notion is that these generators will be called from a root level in code and that the result of one or more generator will be feed into the next generator.
Figure~\ref{fig:generatorexamples} shows three of the different generators and how they can be visualized on a 2D plane. 

\begin{center}
  \begin{figure}[H]
    \begin{center}
      \begin{table}[H]
        \begin{tabular}{lllll}
          \textit{User parameters}              & $\rightarrow$ & \textbf{TerrainGenerator}    & $\rightarrow$ & \textit{Terrain}        \\
          \textit{Terrain, Population markers}  & $\rightarrow$ & \textbf{PopulationGenerator} & $\rightarrow$ & \textit{Population map} \\
          \textit{Terrain, Population map}      & $\rightarrow$ & \textbf{RoadGenerator}       & $\rightarrow$ & \textit{Road network}   \\
          \textit{Road network, Population map} & $\rightarrow$ & \textbf{BlockGenerator}      & $\rightarrow$ & \textit{Block{[}{]}}    \\
          \textit{Block, Population map}        & $\rightarrow$ & \textbf{PlotGenerator}       & $\rightarrow$ & \textit{Plot{[}{]}}     \\
          \textit{Plot}                         & $\rightarrow$ & \textbf{BuildingGenerator}   &               &                         \\
          \textit{Plot}                         & $\rightarrow$ & \textbf{ParkGenerator}       &               &                        
        \end{tabular}
      \end{table}
    \end{center}
    \caption[]{The different proposed generators needed to generate a 3D city}
    \label{fig:generators}
  \end{figure}
\end{center}

\begin{center}
  \begin{figure}[H]
    \centering
    \begin{subfigure}[b]{0.32\textwidth}
      \frame{\includegraphics[width=\linewidth]{figure/method_generation_1.png}}
      \caption{Terrain Generation}
      \label{fig:generatorexamples1}
    \end{subfigure}
    \begin{subfigure}[b]{0.32\textwidth}
      \frame{\includegraphics[width=\linewidth]{figure/method_generation_2.png}}
      \caption{Road Generation}
      \label{fig:generatorexamples2}
    \end{subfigure}
    \begin{subfigure}[b]{0.32\textwidth}
      \frame{\includegraphics[width=\linewidth]{figure/method_generation_3.png}}
      \caption{Block/Plot Generation}
      \label{fig:generatorexamples3}
    \end{subfigure}
    \caption{Visualization of the intended purpose of the Terrain, Road, and Plot/Block generators.}
    \label{fig:generatorexamples}
  \end{figure}
\end{center}
  
\subsection{Terrain Generator}
\begin{center}
  \textit{User parameters} $\rightarrow$ \textbf{TerrainGenerator}  $\rightarrow$ \textit{Terrain}
\end{center}

We will approach this subproblem by first generating a heightmap that represents the hills and the valleys of the terrain.
A heightmap is a bitmap image where each pixel typically store values representing a surface elevation or displacement.
This is often visualized as a grayscale image where the black pixels represents the minimum elevation, and white represents the maximum elevation.
For this project's particular use case the pixel values will store the terrain elevations, which will be rendered as a 3D mesh.
One limitation of using heightmaps is that they can not represent more complex 3D geometry such as caves or overhangs.
However since the main focus of the project is on procedurally generating cities, and not terrain, this is not regarded as a problem.
Figure~\ref{fig:heightmap} illustrates an example heightmap visualized as a grayscale image.

% NOTE(anton): image generated from https://cpetry.github.io/TextureGenerator-Online/
\begin{figure}[h]
  \centering
  \includegraphics[width=0.25\textwidth]{figure/heightmap.png}
  \caption{An example heightmap generated using Perlin Noise.}
  \label{fig:heightmap}
\end{figure}

We plan to generate the heightmap using multiple layers of simplex noise.
Simplex noise is a type of gradient noise commonly used in computer graphics and procedural generation.
The mentioned gradient noise is a type of noise that contains smooth slopes, which causes nearby points to be similar while still remaining random.
The points are then interpolated using a smoothing function to achieve a wave-like texture fitting for representing the terrain.
Each octave, or layer, of simplex noise will represent different levels of terrain granularity.
Higher frequency of noise corresponds to an area on the heightmap where we would like to generate an area with a small, yet intense difference in elevation.
Whereas lower frequencies will correspond to areas of a less intense, but larger difference in elevation forming the contour of the terrain.
This can be used to generate both smooth hills and rocky mountains.

The color of the terrain will be based on the height levels e.g.\ grasslands near sea level, rocky mountain textures for the taller regions, and snow for top peaks of some of the taller mountains.
One approach for achieving this is a technique called Texture splatting that is commonly used for terrain rendering.
This method consists of combining different textures blended together according to an alphamap, also referred to as a weightmap or splatmap.
Figure~\ref{fig:texture-splatting} demonstrates this technique in action.

\begin{figure}[H]
  \centering
  \includegraphics[width=0.4\textwidth]{figure/texture-splatting.png}
  \caption{Example of texture splatting from related work \cite{wiki:texture-splatting-img}}
  \label{fig:texture-splatting}
\end{figure}

Furthermore the user can provide a threshold height for the sea level where any point of the terrain below this threshold value will be filled with ocean. 
This can simply be done with drawing an animated water texture on a plane positioned at said height level. 

As previously mentioned in the problem section, there are some things that could be added to our project in the later stages, if there is time for them. 
% https://arches.liris.cnrs.fr/publications/articles/SIGGRAPH2013_PCG_Terrain.pdf
Something that could be implemented in the later stages of the project would be to add some additional variety to our generated world. 
We could for example add water flowing from mountains that turns into rivers, and similarly we could make these rivers flow into lakes in order to make our terrain look more natural.
These two options being implemented rely primarily on the time which is left after our project satisfies all our primary goals, as they are not the main focus of the project. 
The way rivers can be generated vary greatly, and where they mostly differ is the order of generation around them.
You could choose to generate a heightmap and then try to fit a river basin around it, but some related work believes this is not the way to go. 
The suggested approach from this paper is to combine what the paper refers to as a 'generated terrain slope control' with a 'generated river slope control'. 
After this has been done the river network is generated automatically and the terrain is generated around it following the rules of hydrology. 
~\cite{river_gen}

We believe this will not be our approach as it would require a lot of restructuring of our original project, and not something that would simply be an extension to our final product. 
For now we have only speculated on how we would like to approach this problem, but we have discussed forming rivers using a gradient descent starting from the mountain-tops.

\subsection{Population Generator}
\begin{center}
    \textit{Terrain, Population markers} $\rightarrow$ \textbf{PopulationGenerator} $\rightarrow$ \textit{Population map} 
\end{center}
This generator will from the generated terrain, and the user-specified markers create an intensity map representing the population density for the world.
The intensity map is generated by first applying a few simplex noise layer that represents random distributions of populations throughout the world. 
Then a circle that fades in intensity along its border is added for each population marker. 
The terrain parameter is used to mask away certain locations from being inhabited e.g. oceans and tall mountain peeks.
This will be useful when generating the roads and specifying the blocks from the roads. 
In some places within the world, there is a larger need for higher buildings with many apartments. 
Some other places are more suitable for supermarkets or villas. 
The population map will be the backbone and a large factor for these generations. 

\begin{figure}[h]
  \centering
  \includegraphics[width=0.5\textwidth]{figure/gen_population_map.png}
  \caption{An example of a population map with a Paris city generated within it}
  \label{fig:gen_population_map_example}
\end{figure}

\subsection{Road}
\begin{center}
  \textit{Terrain, Population map} $\rightarrow$ \textbf{RoadGenerator} $\rightarrow$ \textit{Road network} 
\end{center}
Road generation is what will design the entire city, and it will do it by placing different kinds of roads.
We will approach this generation by first allowing the user to decide where city generation shall commence by giving them the option to place a population markers for the roads to be constructed.
The size of the city generated is defined by the user, maximum being the size of the available landscape.
Cities far away from each other will be connected by highways, and there can exist a randomly generated 

\subsubsection{Plotting out cities}
The user should be able to specify exactly where cities should begin their generation sequence.
They will do this via population markers.
Each city has a type which describes the general appearance and layout of the city.
Some examples of city type generation strategies are;
\begin{itemize}
  \item Paris strategy, which would generate circular cities with roads extending from the middle and outwards.
  \item Manhattan strategy, which consists of lots of straight roads going through the city perpendicular to other roads, creating a grid-like appearance.
  \item Chaotic strategy, completely random with lots of turning, no structure whatsoever.
\end{itemize}

Multiple cities can be plotted out at once, and generally when a road leaves the bounds of the city it converts to a highway which aims to connect to other denser areas.

\subsubsection{Main road generation}
Main roads, and roads in general, are generated by what is called \textit{Agents}.
These only have one job, and that is to walk arond the landscape and leave roads behind them.
The agents generate the main roads depending on if it is inside a city, and what type of city it is.
All agents have a strategy that they will follow, instructing them how they should move around.
Agent strategies also tells the agent when it should terminate.

This type of generation results in a lot of flexibility, because each agent can be instructed to walk around differently.
For Paris strategy, some agents are configured to walk around in circles around the center point and others are configured to walk roughly straight outwards from the center.
For Manhattan strategy, the agents simply create completely straight roads, some agents are rotated by 90 degrees to create the perpendicular roads.

This also means agents can decide to switch strategy depending on some criteria, for example when an agent leaves the bounds of a city, it can change its strategy to a highway-type strategy.

Highways are a type of main road, and agents that have a strategy that creates highways should attempt to follow the population map towards higher density areas.
This helps cities to connect to eachother, because highways generally do not turn around as much as other main roads and it will eventually find a higher density area in the population map to go towards.

However, higher density areas does not mean there is always a city located there, it might find an area without a city.
The idea is that the highway strategy can also create new cities or villages if it comes across a very dense area and has not connected to an existing city.
There are cases where a city can be created right next to another city, but that would just result in both cities merging into one.

\subsubsection{Street generation}
While generating main roads, new agents are created which branch off the main road to create streets.
These are generally straight and grid-like, to mimic the patterns found in most neighborhoods.
However, different kind of street strategies can be created to mimic other types of neighborhoods.
To ensure that streets do not cut off main roads, these agents are given a lower priority which ensures that they are always created after main roads are finished generating.
The agents are then prioritized according to the order they were created, which means an agent will generate an entire neighborhood before the next street agent begins.
This creates clearer patterns around the city with less agents competing over the same area.

Agents with a street strategy will always look up the population map before placing a road or branching.
Higher density means a higher chance of branching, creating denser neighborhoods near the center of the city.
An agent that tries to travel into a place with too little of a population density will simply terminate without creating a road. 

\subsection{Block Generator}
\begin{center}
    \textit{Road network, Population map} $\rightarrow$ \textbf{BlockGenerator} $\rightarrow$ \textit{Block{[}{]}}
\end{center}
This generator will analyze the road network and extract areas that are suitable for building city blocks on.
We call these areas \textit{Blocks} and the formal definition can be found in the glossary.
The population map will be used to determine how densely built the \textit{Block} is expected to be.

An example of block generation can be found in Figure~\ref{fig:generatorexamples3}.
The small black roads inside the center area are streets, and the blue lines represent \textit{Plots}.
The yellow, red, and orange areas represent the three \textit{Blocks} that have been generated by the Block Generator.
In this case, the city blocks are completely enclosed within roads, but this is not necessary.
For example, if we imagine a single straight road, then blocks could be generated alongside it without being enclosed by roads.

This generator will be comprised of two algorithms.
The first will extract all areas enclosed by roads where at least one road is a street.
This can be implemented with a Minimum Cycle Basis (MCB) algorithm, which finds the shortest cycles in an undirected graph.
However, this is an NP-hard problem, making it completely unsuitable for generating large road networks.
Fortunately, our graph is restricted to a 3D space.
This restriction lets us traverse the graph by always following the rightmost edge, which results in a polynomial solution.
The 2D version of this idea is discussed in detail in a blog post written by H. Petovan \cite{mcb}.
The second algorithm will try to attach arbitrary \textit{Blocks} to roads where the population is high compared to the number of existing city blocks.
This ensures that \textit{Blocks} are also generated in areas that are not completely enclosed.
This algorithm could potentially be implemented by trial and error of trying to expand blocks from the roads.
\subsubsection{Plot}
\begin{center}
    \textit{Block} $\rightarrow$ \textbf{PlotGenerator} $\rightarrow$ \textit{Plot{[}{]}}
\end{center}
\subsection{Building}
\begin{center}    
    \textit{Plot} $\rightarrow$ \textbf{BuildingGenerator}
\end{center}

\subsubsection{ParkGenerator}
\begin{center}
    \textit{Plot} $\rightarrow$ \textbf{ParkGenerator}                        
\end{center}

