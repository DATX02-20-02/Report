\section{Purpose}
There is a demand within the modern game industry to create large, impressive and unique environments for their player base to explore.
However, this is costly and time consuming to build manually.
But if it is possible to create hills and caves using algorithms as Mojang does in Minecraft, why would it not be possible to generate whole 3D cities?

The purpose of this project is to, through a selection of various PCG techniques, investigate a suitable approach for the generation of a 3D City.
The program will be a desktop application that can generate large cities for use within virtual environments such as games, film and modeling.
The program should be suitable in the case where a great variation of interesting cities is needed with the goal to reduce the amount of time spent on manual editing.

As stated above the generation will be done using a variety of different PCG techniques. It is important to note that the focus of this project is not to invent new PCG algorithms, but rather to make efficient use of already existing ones. 

The reasoning behind this project is to create something which could be put to use in what is believed to be a very rapidly emerging market. And also to acquire some knowledge in a field of work that is very likely to become the norm in not only game-design, but really any field where modeling needs to be done on a greater scale. 

Another area of interest would be to investigate how well these different algorithms behave
depending on which part of a city they are tasked with generating. For example, one road generating algorithm may do well at generating roads within a suburban neighbourhood, but not at generating highways. The goal here is to make sure that each algorithm we make use of is performing whatever task it is best suited for. This is how we believe we will be able to generate cities that are both realistic, and aesthetically pleasing. 

