\section{Population Generation}

When the terrain has been generated, the PopulationGenerator gives the user a visual tool to place a city marker.
When one or more markers are placed, they will remain in their designated positions until interacted with by the user, or when the user presses the Generate Roads button.

A placed city marker can be interacted with by hovering over it, which will turn it red.
Clicking on it will make it yellow, which shows that it is selected. 

\begin{figure}[h!]
  \centering

  \includegraphics[width=0.8\textwidth]{figure/citymarkers.png}
  \caption{The city marker tool visuals. }

  \label{fig:citymarkers}
\end{figure}

The selected city marker has multiple options:

It can be moved by clicking again and dragging the marker around,
it can be resized by scrolling up and down, increasing and decreasing the effective radius of the circle,
and it can be deleted by pressing the Delete key on the keyboard.

Road generation begins after pressing the generate roads button.
Before the roads are generated, a population density map is generated which covers the entire terrain.
Then, the city marker amplifiers add density to the base noise generated in the spots where it was placed.

Roads and highways will tend towards densely populated areas, and the size of buildings will vary depending on how intense the density is.
