\section{Introduction}
The problem of this report revolves around content generation within (predominantly) video games and movie creation. As the modern day computer technology gains increased memory space and performance, the capacity for content in video games and similar fields grows simultaneously. However, programmers' capability to fill this content space remains stagnant.

A prominent example of games that has mastered this problem is Minecraft. Minecraft is a computer game in which the player can explore a pseudo infinite, unique world that is generated algorithmically. This algorithm is built upon an exciting field within computer science called Procedural Content Generation (PCG). 

There is a demand within he modern game industry to create large, impressive and unique environments for their player base to explore. Procedural Content Generation is an option for all game companies (especially small companies on the market) that want to create games that accommodate brimful content (such as GTA, the Fallout franchise, etc.), as it is immensely time consuming to design by hand which in turn is very expensive. So, by inspiration from the cave and hill generation within Minecraft, this project will attempt to create whole 3D cities using an algorithm's automatic generation. In theory, such an automatic city generation could potentially drastically speed up the process it takes to create content for games that want to include a setting within the modern day world.

Worth mentioning is that PCG is not limited to world generation. In this article, PCG will be known as Procedural City Generation, but the concept of Procedural Content Generation is far more general. Other applications include level generation in Diablo, weapons in Borderlands, various creatures encountered within Spore, and the forementioned Minecraft cave systems. The PCG family of algorithms have a great advantage, in that they can generate a whole spectrum of variations of a specific concept so that one would not have to design it themselves.
