\section{Introduction}
Minecraft is a well-known adventure and creative computer game where the player can explore a pseudo infinite, unique world that is generated algorithmically.
This algorithm is built upon an exciting field within computer science called Procedural Content Generation (PCG).
This field has seen many other applications, such as level generation in Diablo, weapons in Borderlands and the various creatures encountered within Spore.
The PCG family of algorithms have a great advantage, in that they can generate a whole spectrum of variations of a specific concept so that one would not have to design it themselves.

In the early days computer games adopted PCG in order to workaround the limitations of hardware.
Today we use in order to achieve labor time from designers and programmers.

As the modern day computer technology gains increased memory space and performance, the capacity for content in video games and similar fields grows simultaneously.
However, the capability for programmers to create this content remains stagnant.
This is a problem for game companies who wants to create games that accommodate brimful content (such as GTA, the Fallout franchise, etc.), as it is immensely time consuming to design, which in turn is very expensive.
This specifically applies for creating world generation, and one such generation is city generation. 

In theory, an automatic city generation could potentially speed up the process it takes to create content for games that want to include a setting within the modern day world.
Today there exists a few games that has done this at great costs, but with extreme success.
The GTA franchise has been very successful, and is known for its modern day open world style.

