\chapter{Introduction}

% Why was the study undertaken i.e. why was this paper written?
% Proper acknowledgements of previous works
% Focus on thesis question(s)
% All cited sources should be directly relevant to the goals of the thesis
% Explain the scope. What will and what will not be included.
% Verbal road map of what's ahead
% Where do old work end and where do our work begin?


% The problem PCG solves.
In conjunction with increasing performance, memory and storage available on modern computers, the capacity for large amounts of varied content in games, movies, and other digital media grows respectively.
However, the capability to manually produce interesting content that can fully utilize this potential remains a slow and expensive process.
Consequently, many companies have decided to leverage the use of algorithms to partly, and sometimes completely, automate the production of such digital content.
These algorithms are part of an exciting research field within computer science called Procedural Content Generation (PCG).

This thesis explores the usage of such PCG algorithms in order to find a suitable approach for procedural generation of modern 3D cities.
The investigation is conducted through the development of a user-friendly desktop application that can interactively generate and export models of 3D cities.
These models can then be used in third-party modeling software and game engines.
Thus, with this report we contribute with insight into various techniques applicable to city generation, as well as an open-source, MIT licensed application called \textit{CityCraft} that demonstrates such generation.

This chapter introduces the project, its goals, and some related works that have been previously done in the field of city generation.
Thereafter, follows a background chapter about the applications of PCG, and a theory chapter covering many of the fundamental techniques considered.
Finally, the project's methods and results are presented, followed up by a discussion chapter and some notes for future work.

\newpage

% Why?
\section{Purpose}

% Open-source MIT
% user-friendly standalone application
% fast and easy (generate a custom city in less than 1 minute on modern laptop)
% must work with third-party software such as Blender and Unity

Cities are complex and massive architectural infrastructures that appear all over the world, and a significant portion of the world population lives inside them.
They are also a frequent scenery in digital media, such as advertisement, video games, and film.
Unfortunately, the creation of digital cities can be a highly time-consuming effort if done by hand.
Ideally, this task could be automated such that studios of any size could produce impressive cities by only specifying some rough design parameters.
The purpose of this project is to contribute towards such an ideal.

Formally, the purpose of this project is to research and combine various PCG techniques in order to discover and demonstrate a suitable approach for generating modern 3D cities. 
The intention is to demonstrate this approach by implementing and releasing a standalone city generation software.
This software should be free, open-source, and accessible to anyone in the industry of digital media.

The significance of such software compares to SpeedTree \cite{speedtree}, which revolutionized the process of creating realistic forests.
Of course, such an ambition is beyond realistic for the scope of this project, and that is why constraints have to be made.
% What and what not? I.e., what exactly?
\section{Goal and Scope}
\label{section:goal_and_scope}

The goal of this project is to develop a user-friendly application that can generate modern 3D cities in real-time and export them in a standard 3D file format such as \textit{.fbx}.
\textit{User-friendly} denotes that users should not need any technical expertise or coding experience to fully utilize the application.
Furthermore, \textit{modern cities} will be defined as cities that resemble present industrialized cities such as New York, Paris, San Francisco, and Tokyo (see Figure~\ref{fig:ModernCities}).
The intention is not to perfectly replicate these cities, but rather to draw inspiration from them.

\begin{figure}[h!]
  \centering

  \begin{subfigure}[b]{0.56\textwidth}
    \frame{\includegraphics[width=\textwidth]{figure/modern-city-manhattan.jpg}}
    \caption{Manhattan, New York \cite{manhattan_img}.}
  \end{subfigure}
  \quad
  \begin{subfigure}[b]{0.395\textwidth}
    \frame{\includegraphics[width=\textwidth]{figure/modern-city-paris.jpg}}
    \caption{Paris \cite{paris_img}.}
  \end{subfigure}

  \caption{Two examples of modern cities considered in this project.}
  \label{fig:ModernCities}
\end{figure}

To clarify, suburban and rural areas are also included as part of such modern cities.

The process of generating cities should be effortless both in the sense that minimal configuration should be required, and that generation of multiple cities should be feasible within a minute from application start when running on a modern laptop.
With that said, the visual quality of models is not of immediate concern for this work.
The intention is to demonstrate a proof of concept of how PCG algorithms can be used to leverage the effort required to build cities, not to produce production-ready software.
Consequently, the use of free textures and models is encouraged in order to prioritize the development of algorithms.
Additionally, any specific art style such as low poly \cite{lowpoly_wiki} and voxel graphics \cite{voxels_wiki} is not pursued.

The generation of roads and buildings will be of primary focus, as these are considered the core features of a city.
Surrounding terrain also needs to be generated, mainly to form natural settings which the city infrastructures need to adjust after.
Accordingly, the terrain itself does not need any significant level of detail.
Other metropolis aspects such as sidewalks, parks and parking lots are also intended to be generated, albeit with less variation.

In this contribution, generated cities will be restricted to static models in order to support integration with a wider array of third-party software.
Therefore, dynamic content such as simulation of road traffic, day-night cycles, and pedestrians are all considered out of scope.
The interior of buildings is also considered out of scope since end-users are expected to desire more control over such content than what this project's time frame can support.

As a consequence of the random and visual aspects of PCG, it is difficult to express a precise definition that can measure whether this project's final results should be considered successful or not.
Nevertheless, the following list has been constructed as a modest checklist to capture and discuss the quality of the resulting application.

\vspace{-0.5cm} % match spacing of easylist
\begin{itemize}
  \item[\textbf{Q1:}] Do models correctly integrate with third-party software such as Blender \cite{blender}?
  \item[\textbf{Q2:}] How well is the codebase structured for replacement and expansion of features?
  \item[\textbf{Q3:}] How much notable variety is there in the generated content?
  \item[\textbf{Q4:}] To what degree do the generated cities resemble real-world cities?
  \item[\textbf{Q5}:] How much control do the users have over the generation?
  \item[\textbf{Q6}:] Are the cities suitable for use in digital media such as games and film?
  \item[\textbf{Q7}:] Are users without technical expertise able to correctly use the application?
  \item[\textbf{Q8}:] Are there any known bugs or crashes in the application, or any visual artifacts in the generated models?
\end{itemize}

% Is this reasonable?
\section{Social and Ethical Aspects}

% Accuracy to real-life and maintain culture. Representing all parts is not focus either.
% What have others done?
\subsection{Related Works}
% Add text here

\subsubsection{Road generation}
Road generation can be accomplished in several methods, some generate more realistic road networks while others are used for cities that do not resemble the real world.
Some methods are more suited for simulation approaches where events occur in realtime, while others are more suited for actual city generation for game development or modeling.

Introversion Software has developed a city generator for quite some time, and it does generate quite realistic results.~\cite{citygen_subversion}.
An attempt to recreate the road generation algorithm was made by Tobias Mansfield-Williams~\cite{citygen_tobias}, and he claims that they followed the paper \textit{Procedural Modeling of Cities}, however, it is difficult to find where the developers stated that fact.
The algorithm that he implemented in his own is using an L-system according to the paper.

Furthermore, related work using the same paper was also developed~\cite{citygen_robin} by Robin, which is also using L-system for the roads. This one, while using roughly the same method, generates a different structure of the road network. It has further restrictions such as restrictions on street generation that prevents them from generating in certain places using the population map as reference.
Robin has also written a paper~\cite{citygen_robin_paper} describing many different types of algorithms and their advantages and disadvantages, and he chose to stick with an L-system, an approach similar to what Introversion Software did.

So far, these methods are based on the same paper with some kind of implementation of an L-system.
However, the algorithm described in the paper \textit{Procedural City Modeling}~\cite{citygen_lechner} is using an agent-based simulation approach. They are builders which place roads around the world based on some criteria. Agents are given different behaviors depending on what structure they should create.
With this particular implementation, the agents only generate roads and there is no classification of which type of road it is, so it is a very simple algorithm that could potentially be extended upon.
