\chapter*{Abstract}

This paper investigates how techniques in Procedural Content Generation (PCG) can be combined with computer graphics theory to generate digital 3D cities suitable for use in media such as advertisements, movies, and games.
This research was conducted to discover ways to alleviate the otherwise time-consuming process of manually modeling such cities.
The investigation analyzes previous work and existing PCG techniques in order to propose a new system capable of generating modern cities from scratch. 
The validity of this system was then demonstrated through the implementation of a city generation software.

As a result of this research, we contribute with \textit{CityCraft}; a free, open-source, MIT licensed desktop application that interactively generates modern cities in real-time.
The generated cities can be exported as 3D models into the royalty-free file format glTF, making them compatible with a wide range of third-party tools for further refinements.
Although of less quality than manually modeled ones, the generated cities are produced in a fraction of the time, and the application itself requires no modeling or programming expertise to be used efficiently.

The realization of CityCraft demonstrates the potential of combining PCG and computer graphics to automate the process of modeling cities.
It also provides insight to how specific techniques, such as Agent-based road generation, L-system-based buildings, and Level of Detail (LOD), can be integrated to achieve performant generation.
The hope with this contribution is that CityCraft can act as a useful stepping stone for further city generation research within the open-source domain.