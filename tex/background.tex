\section{Background}
The problem of this report revolves around content generation within (predominantly) video games and movie creation. 
As the modern day computer technology gains increased memory space and performance, the capacity for content in video games and similar fields grows simultaneously. 
However, programmers' capability to fill this content space remains stagnant. 
Because of this, many major companies make use of algorithms  built upon an exciting field within computer science called Procedural Content Generation (PCG). 

PCG is not something that is new to the industry, but the way it is being utilized is.
In the past PCG was primarily used when creating computer games  in order to workaround the limitations of hardware.
Today however, it is mainly used due to how time-efficient it is. 
With the use of PCG, companies can produce game- or movie environments in hours which would take designers days if not weeks to create manually.

There is a demand within the modern game industry to create large, impressive and unique environments for their player base to explore. 
PCG is an option for all game companies (especially small companies on the market) that want to create games that accommodate brimful content (such as GTA, the Fallout franchise, etc.), as it is immensely time consuming to design by hand which in turn is very expensive. 
By inspiration from the cave and hill generation within Minecraft, this project will attempt to create whole 3D cities using an algorithm's automatic generation. 

Worth mentioning is that PCG has far more applications than city generation.
Other applications include level generation in Diablo, weapons in Borderlands, various creatures encountered within Spore, and the forementioned Minecraft cave systems. 
A key question to be asked is what the actual benefits are of using PCG over manual design. 
One could argue that in comparison to a designer , the PCG family of algorithms have a great advantage. This due to the fact that they can generate a whole spectrum of variations of a specific concept, and do so in only a fraction of the time in which a designer could. 

Today there are few companies who creates content for games that want to include a setting within the modern day world. 
The few who have done it have done so at great costs, but with extreme success.
The GTA franchise has been very successful, and is known for its modern day open world style. 
It is most likely only a matter of time before many other games adapt to this as successfully as GTA has done. 
In other words, it may not be long until PCG is considered one of the key factors in modern day game design.

