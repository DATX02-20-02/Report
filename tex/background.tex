\section{Background}
The problem of this report revolves around content generation within (predominantly) video games and movie creation. 
As the modern day computer technology gains increased memory space and performance, the capacity for content in video games and similar fields grows simultaneously. 
However, designers' capability to create the content by hand remains somewhat stagnant.
Because of this, many major companies make use of algorithms built upon an exciting field within computer science called Procedural Content Generation (PCG). 

PCG is not something that is new to the industry, but the way it is being utilized is.
In the past PCG was primarily used when creating computer games  in order to workaround the limitations of hardware.
Today however, it is mainly used due to how time-efficient it is. 
With the use of PCG, companies can produce impressive environments in hours which would take designers days if not weeks to create manually.

There is a demand within the modern game industry to create large, impressive and unique environments for their player base to explore. 
PCG is an option for all game companies (especially small companies on the market) that want to create games that accommodate brimful content (such as GTA, the Fallout franchise, etc.), as it is immensely time consuming to design by hand which in turn is very expensive. 
By inspiration from the cave and hill generation within Minecraft, this project will attempt to create whole 3D cities using procedural generation. 

Worth mentioning is that PCG has far more applications than city generation.
Other applications include level generation in Diablo, weapons in Borderlands, and all the creatures encountered within Spore.
A key question to be asked is what the actual benefits are of using PCG over manual design. 
One could argue that in comparison to a designer, the PCG family of algorithms have a great advantage. This due to the fact that they can generate a whole spectrum of variations of a specific concept, and do so in only a fraction of the time in which a designer could. 

Procedural generation can be used to generate pretty much any virtual content you could imagine. 
Our project strives to present a possible solution for the procedural generation of cities, in our opinion, one of the vital elements of any generated world. 






