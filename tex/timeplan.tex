\section{Time plan}
% TODO(anton):
% - Decide on suitable iterations
% - Include a section for administrative tasks i.e. Project launch, CI etc.
% - Road Genration Iteration 0: Generate mesh from bézier curve? This can be developed along side the other iterations. doesn't fit my stupid iteration model.
% - include GANTT chart, in the meanwhile it is available in Google Drive

A time plan has been devised to aid the project in staying in phase and keeping the project realistic in relation to the time restriction of the project.
This includes time management of administrative tasks such as the project launch, the developing of the end product as well as the final report.

\subsection{Development of end product}
The three identified subproblems, described in section~\ref{sec:methods}, facilitates a natural division of work since each of the subtasks can almost entirely be developed in isolation.
For instance the road generation can start off by only considering a strictly flat terrain.
Each subproblem is therefore further broken down into different conceptual iterations so that the different systems can be developed simultaneously with as few dependency conflicts as possible.

Each iteration of a subproblem will start off as an isolated simple solution and increase in complexity and coupling with the other subsystems as they are being developed.
This approach will aid us work in a cost-effective manor when developing the end product.

Once all subtasks have undergone all iterations, our efforts will be targeted at fine-tuning the final solution as well as implementing as many of our stretch goals as possible.

\subsubsection{Landscape Generation}
\begin{description}
  \item[Iteration 1] Generate a simple height-map terrain mesh.
  \item[Iteration 2] Populate the terrain with lakes, seas and foliage.
\end{description}

\subsubsection{Road Generation}
\begin{description}
  \item[iteration 0] Convert bezier curve into a road mesh. % this can be developed alongside iteration 1 ...
  \item[Iteration 1] Procedurally generate a convincing model of a road-network on a 2D plane.
  \item[iteration 2] Convert the conceptual road-network into a road mesh.
  \item[Iteration 3] When generating a road from point A to point B, take the height difference in the underlying terrain into account in order to create a natural looking path.
\end{description}

\subsubsection{Building Generation}
\begin{description}
  \item[Iteration 1] Procedurally generate simple block buildings. (Skyscraper, villa, factory?).
  \item[Iteration 2] Given a closed road cycle populate it with suitably placed buildings.
\end{description}


\subsection{Final report}
When it comes to the final written report we are expected to continuously work on the final report during the development.
However we have also set a three week period strictly dedicated to finalizing the report.
At that point we expect the algorithm to be complete as to prioritize the final report.

\subsection{Project plan chart}
To get an overview of our time plan we have decided to include a GANTT chart that roughly describe the expected time of the different tasks.

