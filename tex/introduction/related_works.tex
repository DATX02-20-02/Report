\subsection{Related Works}
A lot of work has been done in the field of procedural city generation and the results varies significantly.
Usually when city generators are created, they have a specific goal in mind.
It could be to generate realistic cities that resemble the real world or to create interesting cities in games.

In the application CityGen~\cite{citygen}, the focus is placed on the generation of smaller roads and buildings, and it has different types of layouts for the building generation.

Introversion Software has developed a city generator for quite some time, and it does generate quite realistic results.~\cite{citygen_subversion}. However, even though it is quite realistic, they chose to use a futuristic style instead.


% Add text here

\subsubsection{Road generation}
% Road generation can be accomplished in several methods, some generate more realistic road networks while others are used for cities that do not resemble the real world.
% Some methods are more suited for simulation approaches where events occur in realtime, while others are more suited for actual city generation for game development or modeling.


An attempt to recreate the road generation algorithm of the Introversion Software city generator~\cite{citygen_subversion} was made by Tobias Mansfield-Williams~\cite{citygen_tobias}, and he claims that they followed the paper \textit{Procedural Modeling of Cities}, however, it is difficult to find where the developers stated that fact.
The algorithm that he implemented in his own is using an L-system according to the paper.

Furthermore, related work using the same paper was also developed~\cite{citygen_robin} by Robin, which is also using L-system for the roads. This one, while using roughly the same method, generates a different structure of the road network. It has further restrictions such as restrictions on street generation that prevents them from generating in certain places using the population map as reference.
Robin has also written a paper~\cite{citygen_robin_paper} describing many different types of algorithms and their advantages and disadvantages, and he chose to stick with an L-system, an approach similar to what Introversion Software did.

So far, these methods are based on the same paper with some kind of implementation of an L-system.
However, the algorithm described in the paper \textit{Procedural City Modeling}~\cite{citygen_lechner} is using an agent-based simulation approach. They are builders which place roads around the world based on some criteria. Agents are given different behaviors depending on what structure they should create.
With this particular implementation, the agents only generate roads and there is no classification of which type of road it is, so it is a very simple algorithm that could potentially be extended upon.
