\subsection{Related Works}
A lot of work has been done in the field of procedural city generation and the results vary significantly.
Usually, when programmers decide to build a generator for cities they have a well-defined idea for what their final product should look like.
It could be to generate cities that resemble the real world, or it could be to create interesting cities in games.

In the application CityGen~\cite{citygen}, the focus is placed on the generation of smaller roads and buildings, and it has different types of layouts for the building generation.

About a decade ago, Introversion Software developed a city generator that displayed some impressive use of procedural generation techniques.~\cite{citygen_subversion}. 
Unfortunately, this project was discontinued leaving only a few demos behind which can be found online.
Even though their project was never finished their demos can still be used as an inspiration for new developers seeking to create a model for the generation of cities.
 
% Add text here

\subsubsection{Road Generation}
% Road generation can be accomplished in several methods, some generate more realistic road networks while others are used for cities that do not resemble the real world.
% Some methods are more suited for simulation approaches where events occur in realtime, while others are more suited for actual city generation for game development or modeling.
An attempt to recreate the road generation algorithm of the Introversion Software city generator~\cite{citygen_subversion} was made by Tobias Mansfield-Williams~\cite{citygen_tobias}. 
Tobias claims that Introversion followed the paper \textit{Procedural Modeling of Cities}, however, it is unknown whether the development team has ever confirmed this statement. 
The algorithm that Tobias implemented is using an L-system and he was also able to produce some impressive results.

Furthermore, related work using the same paper,  \textit{Procedural Modeling of Cities} was also developed~\cite{citygen_robin} by Robin. 
This solution also uses an L-system for the generation of roads. 
While using roughly the same method, this road generator has further restrictions on the street generation that prevents it from generating roads in certain places using a population map as reference.
Robin has also written a paper~\cite{citygen_robin_paper} describing many different types of algorithms and their advantages and disadvantages. 
For his design, Robin chose to stick with an L-system, an approach similar to what Introversion Software did. 

All of the projects mentioned above have to some extent made use of L-systems.
However, the algorithm described in the paper \textit{Procedural City Modeling}~\cite{citygen_lechner} is using an agent-based simulation approach.
Agents are builders which place roads around the world, their behavior differs depending on what structures they are tasked with creating.
With this particular implementation, the agents only generate roads and there is no classification of which type of road it is, so it is a very simple algorithm that could potentially be extended upon.
