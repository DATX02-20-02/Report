\section{Social and Ethical Aspects}

When conducting research it is vital to recognize the social and ethical implications that it might have.
This subchapter covers the major such aspects that were considered throughout this research.

The first aspect considers automation, which is a type of double-edged sword.
On one hand, the development of a city generator could lead to decreased production time and cost for game and film studios alike.
It might particularly help small production studios since they would be able to meet user expectations with less effort, and therefore better compete with the productions of larger studios.
However, this level of automation also has the potential to cause unemployment, especially in major productions where the design of large scenery might involve hundreds of employees.

Unemployment due to automation is a report on its own, but in the case of PCG, it might lead to a shift from designers towards programmers rather than job loss.
With that said, these two professions typically do not overlap in skillset.
According to game developer Scott Beca, attempting to replace level designers with PCG algorithms is ``a dangerous road to go down'' \cite{gamasutra}.
He reckons that, although significant workload traditionally done by designers might transfer to programmers, designers will still be in demand to polish and configure the building blocks used by the PCG algorithms.
Our belief is that such polish applies to the whole generation process, and therefore hope to empower designers by making the application user-friendly.

The second major aspect considers the representation of real-world cities.
If the generated cities fail to capture the various facets that compose real cities, then the generated models might pose a misleading, falsified or biased distortion of reality.
For instance, transportation facilities, history, culture, religion, poverty, politics, authorities and social injustice are all dimensions that can shape the appearance of a city.
As an example, if the generated cities do not include public transportation, does that imply that the application encourages transportation by other means?

Ultimately, a scope will have to be made, and justifying the breadth or depth of real cities within this scope will not be feasible.
Our intention is thus to only include a humble subset of the real-world dimensions, and instead encourage the contribution of further improvements by releasing the application as open-source.

To summarize, we conclude that there exist certain ethical aspects involved with the contribution of this report.
However, measures have been taken to reduce the significance of these potential risks, and with that, the proceedings involved in this research have been considered ethical.

\newpage