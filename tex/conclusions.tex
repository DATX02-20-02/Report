\chapter{Conclusions}

% TODO: A setence mentioning the questions and how we matched them (where did we match well and such) 1-2 sentences

% Clearly state the answer to the main research question
This project aimed to explore the usage of PCG algorithms to procedurally generate modern 3D cities.
This work involved research on previous work and existing PCG techniques in order to design and propose a complete system capable of generating such cities.
The system that we proposed was implemented and demonstrated in the form of a desktop application named CityCraft, which combined several techniques such as Agent-based generation, L-systems, and LOD to achieve performant city generation.
With this implementation, we have demonstrated a suitable approach to city generation by combining PCG algorithms with theory from computer graphics.

% Summarize and reflect on the research
Although far from commercial-ready production quality, the realization of CityCraft provides useful insight into various techniques applicable to city generation, how they can be combined, and also provides a step in the right direction for city generation research within the open-source domain.
With glTF compatibility and no technical expertise required to operate, CityCraft can already find usage in the modeling industry.
However, the software likely needs to mature before that is realistic, with the main limitations being texture quality and certain visual artifacts.

% Make recommendations for future work on the topic
At its current state, CityCraft demonstrates the capability of generating all content included within the scope of this project, with no significant drawbacks or limitations identified.
While further improvements could also include the generation of more content and fixing minor bugs, the primary concern of future research is to identify the limitations of the theoretical system that CityCraft implements.
Our work demonstrates the potential of the proposed system, while future work needs to compare this system with other options to truly deem its quality.
It would also be of interest to stress the system, by including more aspects to generate, such as bridges, interiors of buildings, and public transport infrastructure.

% Show what new knowledge you have contributed
In summary, with this report we hope to bring insight into various methods within PCG and computer graphics suitable for the generation of modern 3D cities.
We also propose a city generation system, utilizing a function-based architecture to orchestrate said methods, and release an implementation of it named CityCraft under the MIT license.
With this contribution, we hope to assist the research of city generation within the open-source domain.