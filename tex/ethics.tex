\section{Social and ethical aspects}
This project's societal impact may affect small game design corporations positively.
The decreased cost to create large amounts of content for games has potential to significantly decrease effort required to design games.
In theory, allowing small game-design companies to use this application could potentially help them compete on the market together with well-made games, as some of the expectations can be met with less effort.
Even large companies can profit from this as a grand part of the designing phase gets automated.

On the other hand, the efficiency of the PCG algorithm may cause job loss for certain programmers.
Computers taking jobs from humans is its own report worth of discussion.
In a nutshell however, our opinion is that there will always be work to do even though the algorithm does simplify the generation of content.
Creators would simply have to displace their resources away from content generation, perhaps towards the PCG algorithm itself or some other major part of their project.

Furthermore, the application could indirectly allow game designers to program an infinitely replayable game, which in turn can be addictive.
For example, it could be a small indie game that relies on micro-transactions to get better cities of some kind.
However, the goal is to create a random city that looks pleasing, therefore avoiding infinite generation would directly conflict with the purpose of this project.

Another ethical dilemma consists of how the cities are created.
Some may find the city generated to be inappropriate or mockful towards the city that the generation is based upon,
for instance subway systems and public transport options.
If they aren't included within the city it could give the feeling that the city exclusively use cars as its way of transport, as if cars are a requirement for city centers.
Another feature that may not be included is religious buildings or other buildings of importance.
This is not to say religion does not have a place in the society, rather that such an addition would possibly require a lot of extra work.
Therefore churches, other unique buildings, landmarks, public transport and many other aspects that we haven't mentioned that could exist within a city may most likely not be included.
If one would like to include those aspects by themself they are free to generate and replace/rebuild any part of the generated city.

In summary we conclude that we do acknowledge the ethical aspects that this report may impose, but to include all (or even some) would require a lot of effort to incorporate given the limited time we have.
Therefore our goal is mostly to focus on the digital aspects within the desgin process of citiy generation. 