\section{Social and ethical aspects}
This project's societal impact may affect small game design corporations positively.
The decreased cost to create large amounts of content for games has potential to significantly decrease effort required to design games. In theory, allowing small gaming companies to use this application could help them compete on the market together with well-made games, as some of the expectations can be met more effortlessly. Even large companies can profit from this as a grand part of the designing phase gets automated.

On the other hand, the application could indirectly potentially allow game designers to program an infinitely replayable game, which in turn can be addictive. For example, it could be a small indie game that relies on micro-transactions to get better cities of some kind. However, the goal is to create a random city that looks pleasing, therefore avoiding infinite generation would directly conflict with the purpose of this project.

In summary we conclude that there are not many relevant social aspects to creating a PCG for cities. Its goal is to ease the desgin process of cities, and all it would do is quicken the design process of cities which could have been done without PCG. 